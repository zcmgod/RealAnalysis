\documentclass[a4paper,12pt]{article} % declaration
\usepackage[utf8]{inputenc}
\usepackage{amsmath}
\usepackage{amsthm,accents}
\usepackage{amsfonts}
\usepackage{color}
\usepackage{graphicx}
\usepackage{tikz}
\usepackage{lmodern,bm}
\usepackage{enumitem}

\newtheorem{definition}{Definition}[section]
\newtheorem{theorem}{Theorem}[section]
\newtheorem{proposition}{Proposition}[section]
\newtheorem{lemma}[theorem]{Lemma}
\newtheorem{corollary}[theorem]{Corollary}
\newtheorem{example}{Examples}
\newtheorem*{remark}{Remark}
\title{Definite Integration}
\author{Guoning Wu}
\begin{document}
%\tableofcontents
\setcounter{tocdepth}{2}
%\listoffigures
%\listoftables
\maketitle
\theoremstyle{definition}

\section{Definition of the Integral and Description of the 
Set of Integrable Functions}
\subsection{Introduction}

Suppose a point is moving along the real line, with $s(t)$ being 
its coordinate at time $t$ and $s'(t) = v(t)$ its velocity at the 
same instant $t$. Assume that we know the position $S(t_0)$ of the 
point at time $t_0$ and that we receive information on its velocity.
Having this function, we wish to compute $s(t)$ for any given value 
of time $t > t_0$.

If we assume that the velocity $v(t)$ varies continuously, the 
displacement of the point over small time interval can be computed 
approximately as the product $v(\tau)\Delta t$ of the velocity at an arbitrary 
instant $\tau$ belonging to that time interval and the magnitude $\Delta t$
of the time interval itself. Taking this observation into account, 
we partition the interval $[t_0, t]$ by marking some times $t_i, i=0,1,\cdots,n$
so that $t_0 < t_1 < \cdots < t_n = t$ and so the interval $[t_{i-1}, t_i]$ are small.
Let $\Delta t_i = t_i - t_{i-1}$ and $\tau_i \in [t_{i-1},t_i]$. Then we have the approximation 
equality 
\[
    s(t) - s(t_0) \approx \sum_{i=1}^n v(\tau_i)\Delta t_i
    \]

The approximation will become more precise if we partition the close 
interval into smaller and smaller intervals. Thus we must conclude 
that in the limit as the length $\lambda$ of the largest of these intervals
tends to zero we shall obtain an exact equality
\begin{equation}
    \lim_{\lambda \to 0}\sum_{i=1}^nv(\tau_i)\Delta t_i = s(t) - s(t_0)
\end{equation}

Such sums, called \textbf{Riemann sums}, are encountered in a wide variety of 
situations.

Let us attempt, for example, following Archimedes, to find the area 
under the parabola $y = x^2$ above the closed interval $[0,1]$.
\[
    \lim_{\lambda \to 0}\sum_{i=1}^nf(\xi_i)\Delta x_i = \frac{1}{3}
    \]
\subsection{Definition of the Riemann Integral}
\paragraph{\rm \textbf{a. Partition}}
\begin{definition}
    \normalfont
    A \textbf{partition} $P$ of a closed interval $[a,b], a < b$, is a finite 
    system of points $x_0, x_1, \cdots, x_n$ of the interval such that 
    $a = t_0 < t_1 < \cdots < t_n = b$.
\end{definition}
The intervals $[t_{i-1},t_i], i=1,2,\cdots,n$ are called the intervals of the
partitions $P$. The largest of the lengths of the intervals of the 
partition $P$, denoted $\lambda(P)$, is called the \textbf{mesh} of the partition.

\begin{definition}
    \normalfont
    We speak of a partition with distinguished points $(P,\xi)$ on the 
    closed interval $[a,b]$ if we have a partition $P$ of $[a,b]$ and 
    a point $\xi \in [t_{i-1},t_i]$ has been chosen in each of the intervals 
    of the partition $[x_{i-1},x_i], i=1,2,\cdots,n$.
\end{definition}
We denoted the set of point $(\xi_1,\cdots,\xi_n)$ by the single letter $\xi$.

\paragraph{\rm \textbf{b. A Base in the Set of Partitions}}
In the set $\mathcal{P}$ of partitions with distinguished points on 
a given interval $[a,b]$, we consider the following base $\mathcal{B} = 
\left\{B_d\right\}.$ The element $B_d, d > 0$, of the base $\mathcal{B}$ consists of 
all partitions with distinguished points $(P,\xi)$ on $[a,b]$ for which 
$\lambda(P) < d.$

\paragraph{\rm \textbf{c. Riemann Sums}}
\begin{definition}
    \normalfont
    If a function $f$ is defined on the closed interval $[a,b]$ and 
    $(P,\xi)$ is a partition with distinguished points on this closed 
    interval, the sum 
    \begin{equation}
        \sigma(f;P,\xi) = \sum_{i=1}^nf(\xi_i)\Delta x_i,
    \end{equation}
    where $\Delta x_i = x_i - x_{i-1}$, is the \textbf{Riemann sum} of the function $f$
    corresponding to the partition $(P,\xi)$ with distinguished point on 
    $[a,b]$.
\end{definition}

Thus, when the function $f$ is fixed, the Riemann sum $\sigma(f;P,\xi)$ is a function 
$\Phi(p) = \sigma{f;\sigma}$ on the set $\mathcal{P}$ of all partitions $p = (P,\xi)$
with distinguished point on the closed interval $[a,b]$. Since there 
is a base $\mathcal{B}$ in $\mathcal{P}$, one can ask about the limit 
of the function $\Phi{p}$ over the base.

\paragraph{\rm \textbf{d. The Riemann Integral}}
Let $f$ be a function defined on a closed interval $[a,b]$.
\begin{definition}
    \normalfont
    uran
    The number $I$ is the \textbf{Riemann integral} of the function $f$ on the 
    closed interval $[a,b]$ if for every $\epsilon > 0$ there exists $\delta > 0$ 
    such that 
    \[
        \left\vert I - \sum_{i=1}^nf(\xi_i)\Delta x_i \right \vert < \epsilon
        \]
    for any partition $(P,\xi)$ with distinguished points on $[a,b]$ whose 
    mesh $\lambda(P)$ is less than $\delta$.
    \label{def:def1}
\end{definition}

Since the partition $p = (P,\xi)$ for which $\lambda(P) < \delta$ form the element 
$B_{\delta}$ of the base $\mathcal{B}$ introduced above in the set $\mathcal{P}$
of partitions with distinguished points, the above definition is  
equivalent to 
\[
    I = \lim_{\mathcal{B}}\Phi(p)
    \]

The integral of $f(x)$ over $[a,b]$ is denoted 
\[
    \int_a^bf(x)\;\mathrm{d}x,
    \]
in which the number $a$ and $b$ are called respectively the lower and 
upper limits of integration. The function $f$ is called the integrand, 
$f(x)\mathrm{d}x$ is called the differential form, and $x$  is the variable 
of integration. Thus 
\begin{equation}
    \int_a^bf(x)\;\mathrm{d}x = \lim_{\lambda(P)\to 0}\sum_{i=1}^nf(\xi_i)\Delta x_i
    \label{eq:eq1}
\end{equation}

\begin{definition}
    \normalfont 
    A function $f$ is Riemann integrable on the closed interval $[a,b]$
    if the limit of the Riemann sums in Eq.~\ref{eq:eq1} exists 
    as $\lambda(P)\to 0$(that is, the Riemann integral of $f$ is defined).
\end{definition}

The set of Riemann-integrable functions on a closed interval $[a,b]$
will be denoted $\mathcal{R}[a,b]$.

\subsection{The Set of Integrable Functions}
The integrability or non-integrability of a function $f$ on $[a,b]$
depends on the existence of the limit below 
\[
    \lim_{\lambda(P)\to 0}\sum_{i=1}^nf(\xi_i)\Delta x_i
\]
By the Cauchy criterion, this limit exists if and only if for every 
$\epsilon > 0$ there exists an element $B_{\delta} \in \mathcal{B}$ in the base such that 
\[
    \left\vert \Phi(p') - \Phi(p'')\right\vert < \epsilon
    \]
for any two points $p', p'' \in B_{\delta}$.

In more detailed notation, what has just been said means that for 
any $\epsilon > 0$ there exists $\delta > 0$ such that 
\[
    \left\vert \sigma(f;P',\xi') - \sigma(f;P'',\xi'')\right\vert < \epsilon
    \]
or, what is the same,
\[
    \left\vert \sum_{i=1}^{n'}f(\xi'_i)\Delta x'_i - \sum_{i=1}^{n''}f(\xi''_i)\Delta x''_i\right\vert 
    < \epsilon 
    \]
for any partition $(P',\xi')$ and $(P'',\xi'')$ with distinguished points 
on the interval $[a,b]$ with $\lambda(P') < \delta$ and $\lambda(P'') < \delta$.

\paragraph{\rm \textbf{a. A Necessary Condition for Integrability.}}
\begin{proposition}
    \normalfont
    A necessary condition for a function $f$ defined on a closed interval 
    $[a,b]$ to be Riemann integrable on $[a,b]$ is that $f$ be bounded on 
    $[a,b]$.
\end{proposition}

\paragraph{\rm \textbf{b. A Sufficient Condition for Integrability and the 
Most Important Classes of Integrable Functions}}
We begin with some notation and remarks that will be used in the explanation 
to follow.

We agree that when a partition $P$
\[
    a = x_0 < x_1 < \cdots < x_n = b
    \]
is given on the interval $[a,b]$, we shall use the symbol $\Delta_i$ to 
denote the interval $[x_{i-1}, x_i]$ along with $\Delta x_i$ as a notation 
for the difference $x_i - x_{i-1}$. If a partition $\widetilde{P}$
of the closed interval $[a,b]$ is obtained from a partition $P$
by the jointing new points to $P$, we call $\widetilde{P}$ a 
refinement of $P$. When a refinement $\widetilde{P}$ of a partition 
$P$ is constructed, some of the closed intervals $\Delta_i=[x_{i-1},x_i]$ 
of the partition $P$ themselves undergo partitioning:
\[
    x_{i-1} = x_{i0} < x_{i1} < \cdots < x_{in_i} = x_i.
    \]
\begin{proposition}
    \normalfont
    A sufficient condition for a bounded function $f$ to be integrable 
    on a closed interval $[a,b]$ is that for every $\epsilon > 0$ there exist a 
    number $\delta > 0$ such that 
    \[
        \sum_{i=0}^n\omega(f;\Delta_i)\Delta x_i < \epsilon 
        \]
    for any partition $P$ of $[a,b]$ with mesh $\lambda(P) < \delta$.
\end{proposition}
\begin{proof}
    Let $P$ be a partition of $[a,b]$ and $\widetilde{P}$ a refinement of 
    $P$. Let us estimate the difference between the Riemann sums 
    $\sigma(f;\widetilde{P},\widetilde{\xi}) - \sigma(f;P,\xi)$. Using the notation 
    introduced above, we can write
    \[
        \begin{split}
        \left\vert \sigma(f;\widetilde{P},\widetilde{\xi}) - 
            \sigma(f;P,\xi)\right\vert 
        & = \left\vert \sum_{i=1}^n\sum_{j=1}^{n_j}f(\xi_{ij})\Delta x_{ij} - \sum_{i=1}^nf(\xi_i)\Delta x_i 
            \right\vert \\
            & = \left\vert \sum_{i=1}^n\sum_{j=1}^{n_j}f(\xi_{ij})\Delta x_{ij} - \sum_{i=1}^n\sum_{j=1}^{n_j}f(\xi_i)\Delta x_{ij} 
            \right\vert \\
        & = \left\vert\sum_{i=1}^n\sum_{j=1}^{n_j}(f(\xi_{ij}) - f(\xi_i))\Delta x_{ij}\right\vert 
             \le \sum_{i=1}^{n}\sum_{j=1}^{n_j}\vert f(\xi_{ij}) - f(\xi_i) \vert \Delta x_{ij} \\
        & = \sum_{i=1}^n\sum_{j=1}^{n_j}\omega(f;\Delta_i)\Delta x_{ij} = \sum_{i=1}^n\omega(f;\Delta x_i)\Delta x_i.
        \end{split}
        \]
    It follows from the estimation for the difference of the Riemann 
    sums that if the function satisfies the sufficient condition given 
    in the statement of the proposition, then for each $\epsilon > 0$, we can 
    find $\delta > 0$ such that 
    \[
        \vert \sigma(f;\widetilde{P},\widetilde{\xi}) - \sigma(f;P,\xi) \vert < \frac{\epsilon}{2}
       \]
    Now if $(P',\xi')$ and $(P'',\xi'')$ are arbitrary partitions with distinguished 
    points on $[a,b]$ whose meshes satisfy $\lambda(P') < \delta$ and $\lambda(P'') < \delta$, then,
    by what has been proved, the partition $\widetilde{P} = P'\cup P'' $,
    we have 
    \[
        \left\vert \sigma(f;\widetilde{P},\widetilde{\xi}) - 
        \sigma(f;P',\xi') \right \vert < \frac{\epsilon}{2}
        \]
    \[
        \left\vert \sigma(f;\widetilde{P},\widetilde{\xi}) - 
        \sigma(f;P'',\xi'') \right \vert < \frac{\epsilon}{2}
        \]
    It follows that 
    \[
        \left\vert \sigma(f;P',\xi') - \sigma(f;P'',\xi'') \right\vert < \epsilon
        \]
    provided that $\lambda(P') < \delta, \lambda(P'') < \epsilon$. Therefore, by the Cauchy 
    criterion, the limit of the Riemann sums exists:
    \[
        \lim_{\lambda(P)\to 0} \sum_{i=1}^nf(\xi_i)\Delta x_i,
        \]
    that is $f \in \mathcal{R} [a,b]$.
\end{proof}

\begin{corollary}
    \normalfont
    $\displaystyle \left(f \in C[a,b]\right) \Rightarrow \left(f \in \mathcal{R}[a,b]\right)$, that is, 
    every continuous function on a closed interval is integrable 
    on that close interval.
\end{corollary}

\begin{corollary}
    \normalfont
    If a bounded function $f$ on a closed interval $[a,b]$ is continuous 
    everywhere except at a finite set of points, then $f \in \mathcal{R}[a,b]$.
\end{corollary}

\begin{corollary}
    \normalfont 
    A monotonic function on a closed interval is integrable on that 
    interval.
\end{corollary}

\begin{definition}
    \normalfont 
    Let $f: [a,b]\to \mathbb{R}$ be a real valued function that is defined 
    and bounded on the closed interval $[a,b]$, let $P$ be a partition of 
    $[a,b]$, and let $\Delta_i(i=1,2,\cdots,n)$ be the intervals of the partition $P$.
    Let $\displaystyle m_i = \inf_{x\in \Delta_i}f(x)$ and
    $\displaystyle M_i = \sup_{x \in \Delta_i} f(x), i=1,2,\cdots,n$.
    
    The sums 
    \[
        s(f;P) = \sum_{i=1}^nm_i\Delta x_i 
        \]
    and 
    \[
        S(f;P) = \sum_{i=1}^nM_i\Delta x_i
        \]
    are called respectively the lower and upper Riemann sums of the function 
    $f$ on the interval $[a,b]$ corresponding to the partition $P$ of the 
    interval.
    The sums $s(f;P)$ and $S(f;P)$ are also called the lower and upper 
    \textbf{Darboux} sums corresponding to the partition $P$ of $[a,b]$.
\end{definition}

If $(P,\xi)$ is an artitrary partition with distinguished points on 
$[a,b]$, then obviously 
\begin{equation}
    s(f;P) \le \sigma(f;P,\xi) \le S(f;P)
\end{equation}

\begin{lemma}
    \[
        \begin{split}
            s(f;P) & = \inf_{\xi}\sigma(f;P,\xi)\\
            S(f;P) & = \sup_{\xi}\sigma(f;P,\xi)
        \end{split}
        \]
\end{lemma}
\begin{proposition}
    \normalfont
    A bounded real-valued function $f:[a,b]\to \mathbb{R}$ is 
    Rimann integrable on $[a,b]$ if and only if the following 
    limit exist and are equal to each other:
    \begin{equation}
        \underline{I} = \lim_{\lambda (P)\to 0}s(f;P); 
        \overline{I} = \lim_{\lambda (P)\to 0}S(f;P).
    \end{equation}
    When this happens, the common value $I = \underline{I} 
    = \overline{I}$ is the integral 
    \[
        \int_a^bf(x)\;\mathrm{d}x
        \]
\end{proposition}

\begin{proposition}
    \normalfont 
    A necessary and sufficient condition for a function $f:
    [a,b] \to \mathbb{R}$ defined on a closed interval $[a,b]$ to 
    be \textbf{Riemann integrable} on $[a,b]$ is the following relation:
    \begin{equation}
        \lim_{\lambda(P) \to 0}\sum_{i=1}^n\omega(f;\Delta_i)\Delta x_i = 0
    \end{equation}
\end{proposition}

\paragraph{\rm \textbf{c. The Vector Space $\mathcal{R}[a,b]$}}
\begin{proposition}
    \normalfont
    If $f,g\in \mathcal{R}[a,b]$, then 
    \begin{enumerate}
        \item $(f+g) \in \mathcal{R}[a,b]$;
        \item $\alpha f \in \mathcal{R}[a,b]$, where $\alpha$ is a numerical coefficient;
        \item $\vert f \vert\in \mathcal{R}[a,b] $;
        \item $f\vert_{[c,d]} \in \mathcal{R}[a,b]$ if $[c,d] \subset [a,b]$;
        \item $(f\cdot g) \in \mathcal{R}[a,b]$.
    \end{enumerate}
\end{proposition}

\end{document}
