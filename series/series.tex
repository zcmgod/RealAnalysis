\documentclass[a4paper,12pt]{article}
\usepackage[utf8]{inputenc}
\usepackage{CJKutf8}
\usepackage{multirow}
\usepackage{graphicx}
\usepackage{amsmath}
\usepackage{amsthm}
\usepackage{enumerate}
\usepackage{fancyhdr}
\usepackage{setspace}
\usepackage{enumitem}
\newtheorem{theorem}{定理}
\newtheorem{lemma}{引理}
\newtheorem{definition}{定义}
\newtheorem{example}{例子}
\newtheorem{corollary}{推论}
\newtheorem{remark}{注}
\let\oldref\ref
\renewcommand{\ref}[1]{\rm{(\oldref{#1})}}
\renewcommand{\headrulewidth}{0.4pt}
\renewcommand{\footrulewidth}{0.4pt}

\begin{CJK}{UTF8}{gbsn}
\begin{document}
\title{数项级数}
\author{武国宁}
\date{}
\maketitle

\section{简介}
    早在公元前450年,古希腊有一位名叫Zeno的学者,曾提出若干个在
数学发展史上产生过重大影响的悖论,"Achilles(希腊神话中的英雄)
追赶乌龟"即是其中较为著名的一个。
    
    有限$-->$ 无限。
    
    这种“无限多个数相加,相乘是否一定有意义?若不一定,怎么来判
別,无限多个数相加是否符合交换律,结合律等等。对于无限多个函数,
.......

\section{数项级数的收敛性}
\subsection{数项级数}
    设$x_1, x_2, \cdots, x_n, \cdots$ 是无限可列个实数,则称
\[
        x_1 + x_2 + \cdots + x_n + \cdots
\]
为无穷项级数(简称级数),记为$\sum_{n=1}^{\infty} x_n$,
其中,$\displaystyle x_n$ 称为级数的通项或一般项。

    数项级数的部分和数列${S_n}$定义为:
\[
    \begin{split}
        S_1 &= x_1 \\
        S_2 &= x_1 + x_2 \\
    \vdots\\
        S_n &= x_1 + x_2 + \cdots + x_n
    \end{split}
    \]
\begin{definition}
    如果部分和数列${S_n}$ 收敛于有限数$S$,则称无穷级数
    $\displaystyle \sum_{n=1}^{\infty}$ 收敛,且和为$S$ ,记为:
    \[
        S = \sum_{n=1}^{\infty} x_n
        \]
    如果部分和数列发散,则称无穷级数发散。
\end{definition}
\begin{example}
    设$|q| < 1$, 讨论几何级数(等比级数)
    \[
        \sum_{n=1}^{\infty} q^{n-1} = 1 + q + q^2 + \cdots + q^n + \cdots
        \]
    的敛散性。
\end{example}
\begin{example}
    讨论级数:
    \[
        \sum_{n=1}^{\infty} (-1)^{n-1} = 1 - 1 + 1 + \cdots + 
        (-1)^{n-1} + \cdots
        \]
    的敛散性。
\end{example}
\begin{theorem}
    级数$\displaystyle \sum_{n=1}^{\infty} u_n$ 收敛的充分必要条件为:
    任给$\epsilon>0$,总存在正数$N$,使得当$m > N$ 以及对于任意的正整数$p$,
    都有:
    \[
        \left| u_{m+1} + u_{m+2} + \cdots + u_{m+p} \right| < \epsilon.
        \]
\end{theorem}
\begin{example}
    讨论级数:
    \[
        \sum_{n=1}^{\infty} \frac{1}{n^p} = 1 + \frac{1}{2^p} + \cdots + 
        \frac{1}{n^p} + \cdots
        \]
    的敛散性。
\end{example}

\subsection{级数收敛的基本性质}
\begin{theorem}
    设级数$\displaystyle \sum_{n=1}^{\infty} u_n$ 收敛,则有:
    \[
        \lim_{n \to \infty} u_n = 0
        \]
\end{theorem}
\begin{theorem}
    设级数$\displaystyle \sum_{n=1}^{\infty} u_n$ ,$\displaystyle \sum_{n=1}^{\infty} v_n$ 收敛,
    则对于任意的常数$\alpha, \beta$,级数$\displaystyle \sum_{n=1}^{\infty} \alpha u_n + \beta v_n$ 
    收敛,且有:
    \[
        \displaystyle \sum_{n=1}^{\infty}  \left(\alpha u_n + \beta v_n  \right) 
        =\alpha \sum_{n=1}^{\infty} + \beta \sum_{n=1}^{\infty} v_n
        \]
\end{theorem}
\begin{theorem}
    去掉,增加或改变级数的有限项不影响级数的敛散性。
\end{theorem}
\begin{theorem}
    在收敛级数中任意加括弧,既不改变级数的收敛性,也不改变级数的和。
\end{theorem}

\begin{example}
    讨论级数
    \[
        \frac{1}{\sqrt{2} - 1} - \frac{1}{\sqrt{2} + 1} + 
        \frac{1}{\sqrt{3} - 1} - \frac{1}{\sqrt{3} + 1} + 
        \cdots + \frac{1}{\sqrt{n} - 1} - \frac{1}{\sqrt{n} + 1} + 
        \cdots +
        \]
    的敛散性。
\end{example}

\begin{example}
    讨论级数:
    \[
        \sum_{n=1}^{\infty} \frac{2n - 1}{2^n}
        \]
    的敛散性。
\end{example}
\begin{example}
    讨论级数:
    \[
        \sum_{n=1}^{\infty} \arctan\frac{1}{2n^2}
        \]
    的敛散性。
    提示:
    \[
        \arctan x - \arctan y = \arctan \frac{x - y}{1 + xy}
        \]
    \[
        \arctan \frac{1}{2n^2} = \arctan \frac{1}{2n-1} - 
        \arctan \frac{1}{2n + 1}
        \]
\end{example}

\subsection{作业}
\subsubsection{证明下列级数收敛,并求其和}
\begin{enumerate}[label={\rm(\arabic*)}]
    \item $\displaystyle \left(\frac{1}{2} + \frac{1}{3}\right) + 
        \left(\frac{1}{2^2} + \frac{1}{3^2}\right) + \cdots + 
        \left(\frac{1}{2^n} + \frac{1}{3^n}\right)$
    \item$ \displaystyle \sum_{n=1}^{\infty} \frac{1}{n(n+1)(n+2)}$ 
    \item$ \displaystyle \sum_{n=1}^{\infty} \left(\sqrt{n+2} - 2\sqrt{n+1}
        + \sqrt{n} \right)$ 
\end{enumerate}

\subsubsection{证明题}
证明:若数列$\left\{a_n\right\}$收敛于$a$,则级数$\displaystyle \sum_{n=1}^{\infty} \left( a_n - 
a_{n+1}\right) = a_1 - a$.  
\subsubsection{证明题}
证明:若数列$\left\{b_n\right\}$ 有$ \displaystyle \lim_{n \to \infty} b_n = \infty$, 则:
\begin{enumerate}[label={\rm(\arabic*)}]
    \item 级数$\displaystyle \sum_{n=1}^{\infty} \left(b_{n+1} - b_n\right)$发散;
    \item 当$\displaystyle b_n \ne 0$时,级数$\displaystyle \sum_{n = 1} ^{\infty} 
        \left( \frac{1}{b_n} - \frac{1}{b_{n+1}}\right) = \frac{1}{b_1}$.
\end{enumerate}
\subsubsection{利用上述结果求下列级数的和}
\begin{enumerate}[label={\rm(\arabic*)}]
    \item $\displaystyle \sum_{n=1}^{\infty} \frac{1}{(a+n-1)(a+n)}$
    \item $\displaystyle \sum_{n=1}^{\infty} (-1)^{n+1} \frac{2n+1}{n(n+1)}$
\end{enumerate}

\subsubsection{应用柯西收敛原理证明下列级数的敛散性}
\begin{enumerate}[label={\rm(\arabic*)}]
    \item $\displaystyle \sum_{n=1}^{\infty} \frac{\sin 2^n}{2^n}$
    \item $\displaystyle \sum_{n=1}^{\infty} \frac{1}{\sqrt{n+n^2}}$
\end{enumerate}
 
\section{正向级数}
\subsection{正向级数收敛性的一般判别法则}
    若数项级数的各项符号相同,则称它为同号级数。对于同号级数,
只需研究各项都是正数组成的级数--正项级数。
\begin{theorem}
    正向级数$\displaystyle \sum u_n$ 收敛的充要条件是:部分和数列
    $\displaystyle \left\{S_n\right\}$有界。
\end{theorem}
\begin{theorem}
    设$\displaystyle \sum u_n, \sum v_n$ 是两个正向级数,如果存在
    某个正数$N$ ,对于一切$n > N$,都有:
    \[
        u_n \le v_n
        \]
    则:
    \begin{enumerate}[label={\rm(\arabic*)}]
        \item 若级数$\displaystyle \sum v_n$ 收敛,则级数$\displaystyle \sum u_n$ 收敛;
        \item 若级数$\displaystyle \sum u_n$ 发散,则级数$\displaystyle \sum v_n$ 发散。
    \end{enumerate}
\end{theorem}

\begin{example}
    讨论级数$\displaystyle \sum \frac{1}{n^2 - n + 1}$的敛散性。
\end{example}

\begin{example}
    讨论级数$\displaystyle \sum \sin \frac{\pi}{n}$的敛散性。
\end{example}

\begin{corollary}
    设$\displaystyle \sum u_n, \sum v_n$ 是两个正向级数,如果
    \[
        \lim_{n \to \infty} \frac{u_n}{v_n} = l
        \]
    则:
    \begin{enumerate}[label={\rm(\arabic*)}]
        \item 当$0 < l < \infty$,则级数$\displaystyle \sum v_n, \sum u_n$ 同时收敛或
            同时发散;
        \item 当$l = 0$,级数$\displaystyle \sum u_n$ 收敛时,则级数
            $\displaystyle \sum v_n$ 收敛;
        \item 当$l = +\infty$,级数$\displaystyle \sum u_n$ 发散时,则级数
            $\displaystyle \sum v_n$ 发散;
    \end{enumerate}
\end{corollary}

\begin{example}
    讨论级数$\displaystyle \sum \frac{1}{2^n-1}$的敛散性。
\end{example}

\begin{example}
    讨论级数$\displaystyle \sum \sin \frac{1}{n}$的敛散性。
\end{example}

\begin{example}
    讨论级数$\displaystyle \sum \left(1 - \cos \frac{\pi}{n}\right)$的敛散性。
\end{example}

\subsection{比值判别法和根值判别法}
\begin{theorem}
    设$\displaystyle \sum u_n$ 为正项级数,且存在某正数$N_0$及常数
    $q (0 < q < 1).$ 
    \begin{enumerate}[label={\rm(\arabic*)}]
        \item  对于一切$n > N_0$,成立不等式
            \[
                \frac{u_{n+1}}{u_n} \le q
                \]
        则级数$\displaystyle \sum u_n$ 收敛。
        \item  对于一切$n > N_0$,成立不等式
            \[
                \frac{u_{n+1}}{u_n} \ge 1
                \]
        则级数$\displaystyle \sum u_n$ 发散。
    \end{enumerate}
\end{theorem}

\begin{corollary}
    设$\displaystyle \sum u_n$ 为正项级数,且
    \[
        \lim_{n \to \infty} \frac{u_{n+1}}{u_n} = q
        \]
    则,
    \begin{enumerate}[label={\rm(\arabic*)}]
        \item  当$q < 1$,则级数$\displaystyle \sum u_n$ 收敛。
        \item  当$q > 1$ 或$q = \infty$,则级数$\displaystyle \sum u_n$ 发散。
    \end{enumerate}
\end{corollary}

\begin{example}
    讨论级数$\displaystyle \sum nx^{n-1}(x>0)$ 的敛散性。
\end{example}

\begin{example}
    讨论级数$\displaystyle \sum \frac{n!}{n^n}$ 的敛散性。
\end{example}

\begin{theorem}
    设$\displaystyle \sum u_n$ 为正项级数,且存在某正数$N_0$及常数
    $q (0 < q < 1).$ 
    \begin{enumerate}[label={\rm(\arabic*)}]
        \item  对于一切$n > N_0$,成立不等式
            \[
                \sqrt[n]{u_n} \le q
                \]
        则级数$\displaystyle \sum u_n$ 收敛。
        \item  对于一切$n > N_0$,成立不等式
            \[
                \sqrt[n]{u_n} \ge 1
                \]
        则级数$\displaystyle \sum u_n$ 发散。
    \end{enumerate}
\end{theorem}

\begin{corollary}
    设$\displaystyle \sum u_n$ 为正项级数,且
    \[
        \lim_{n \to \infty} \sqrt[n]{u_n} = q
        \]
    则,
    \begin{enumerate}[label={\rm(\arabic*)}]
        \item  当$q < 1$,则级数$\displaystyle \sum u_n$ 收敛。
        \item  当$q > 1$,则级数$\displaystyle \sum u_n$ 发散。
    \end{enumerate}
\end{corollary}

\begin{example}
    讨论级数$\displaystyle \sum \frac{2 + (-1)^n}{2^n}$ 的敛散性。
\end{example}

\begin{example}
    讨论级数$\displaystyle \sum \frac{x^n}{1 + x^{2n}}$ 的敛散性。
\end{example}

\begin{example}
    讨论下列级数的敛散性$\displaystyle \sum \frac{(n!)^2}{(2n)!}$ ,
    $\displaystyle \sum \frac{n^2}{\left(2 + \frac{1}{n}\right)^n}$
\end{example}

\subsection{积分判别法}
\begin{theorem}
    设 $f$ 为 $[1, +\infty)$ 上非负递减函数,那么正向级数$\sum f(n)$ 
    与反常积分 $\displaystyle \int_1^{+\infty} f(x)\,\mathrm{d}x$
    同时收敛或发散。
\end{theorem}

\begin{example}
    讨论级数$\displaystyle \sum \frac{1}{x^p}$ 的敛散性。
\end{example}

\begin{example}
    讨论下列级数的敛散性$\displaystyle \sum \frac{1}{n\left(\ln n\right)^n}$ ,
    $\displaystyle \sum \frac{1}{n \ln n \left(\ln \ln n\right)^n}$
\end{example}

\subsection{作业}
\subsubsection{判别下列级数的敛散性}
\begin{enumerate}[label={\rm(\arabic*)}]
    \item $\displaystyle \sum \frac{1}{n^2 + a^2}$
    \item $\displaystyle \sum 2^n\frac{\pi}{3^n}$
    \item $\displaystyle \sum \frac{\pi}{n\sqrt[n]n}$
    \item $\displaystyle \sum \frac{(n+1)!}{10^n}$
    \item $\displaystyle \sum \frac{n^2}{2^n}$
\end{enumerate}

\subsubsection{采用积分判别法判别下列级数的敛散性}
\begin{enumerate}[label={\rm(\arabic*)}]
    \item $\displaystyle \sum \frac{1}{n^2 + 1}$
    \item $\displaystyle \sum \frac{n}{n^2 + 1}$
\end{enumerate}

\subsubsection{证明题}
设$\displaystyle a_n \ge 0, n = 1,2, \cdots$ 且$\displaystyle \left\{na_n\right\}$
有界,证明$\displaystyle a_n^2$收敛。 

\section{一般项级数}
\subsection{交错项级数}
若级数的各项符号正负相间,即:
\begin{equation}
    u_1 - u_2 + u_3 - u_4 + \cdots + (-1)^{n-1}u_n + \cdots \left(u_n > 0, n = 1, 2, \cdots\right)
    \label{eq:eq1}
\end{equation}
\begin{theorem}{\rm\textbf{(莱布尼茨判别法)}}
    若交错级数\ref{eq:eq1}满足下述两个条件:
    \begin{enumerate}[label={\rm(\arabic*)}]
        \item 数列$\displaystyle u_n$单调递减;
        \item $\displaystyle \lim_{n \to \infty} u_n = 0.$
    \end{enumerate}
    则级数\ref{eq:eq1}收敛。
\end{theorem}
\begin{proof}
    利用单调有界原理。
\end{proof}

\begin{corollary}
    若级数\ref{eq:eq1}满足莱布尼茨判别法的条件,则级数\ref{eq:eq1}
    的余项估计式为:
    \[
        \left|R_n\right| \le u_{n+1}
        \]
\end{corollary}

\begin{example}
    \begin{enumerate}[label={\rm(\arabic*)}]
        
        \item $\displaystyle 1 - \frac{1}{2} + \frac{1}{3} + \cdots + 
            (-1)^{n-1}\frac{1}{n} + \cdots,$
        \item $\displaystyle 1 - \frac{1}{3!} + \frac{1}{5!} + \cdots + 
            (-1)^{n+1}\frac{1}{2n-1} + \cdots,$
        \item $\displaystyle \frac{1}{10} - \frac{2}{10^2} + \frac{3}{10^3} + \cdots + 
            (-1)^{n-1}\frac{n}{10^n} + \cdots$
    \end{enumerate}
\end{example}

\subsection{绝对收敛级数及其性质}
若级数
\begin{equation}
    u_1 + u_2 + \cdots + u_n + \cdots 
    \label{eq:eq2}
\end{equation}
各项的绝对值组成的级数
\begin{equation}
    \left| u_1 \right| + \left| u_2 \right| + \cdots + \left| u_n \right| + \cdots 
    \label{eq:eq3}
\end{equation}
收敛,则称级数\ref{eq:eq3}为绝对收敛级数。

\begin{theorem}
    绝对收敛的级数一定收敛。
\end{theorem}

\begin{example}
    讨论级数
    \[
        \sum_{n=1}^{\infty} \frac{\alpha^n}{n!} = \alpha + \frac{\alpha^2}{2!} + \cdots +
        \frac{\alpha^n}{n!}
        \]
    的绝对收敛(讨论$\alpha$取值范围)。
\end{example}

\begin{remark}{\rm 级数的重排}
    我们把正整数列:$\displaystyle \left\{1, 2, 3, \cdots, n, \cdots\right\}$
    到它自身的一个映射$\displaystyle f: n \to k(n)$
    称为整整数列的重排,相应的对于数列$\displaystyle \left\{u_n\right\} $
    按照映射$\displaystyle F: u_n \to u_{k(n)}$所得到的数列
    $\displaystyle \left\{u_{k(n)}\right\}$称为原级数的重排。为叙述方便,记
    $\displaystyle v_n = u_{k(n)}$, 即级数$\displaystyle \sum_{n=1}^{\infty} u_{k(n)}$
    写作:
    \begin{equation}
        v_1 + v_2 + \cdots + v_n + \cdots
        \label{eq:eq4}
    \end{equation}
\end{remark}
\begin{theorem}
    级数\ref{eq:eq2}绝对收敛,且和等于$S$,则任意重排后得到的级数
    \ref{eq:eq4}也绝对收敛,且有相同的和数。
\end{theorem}

\begin{remark}
    由条件收敛的级数重排后得到新的级数,即使收敛,也不一定收敛到
    原来的和数。一般的,对于条件收敛的级数,可以通过重排收敛于任
    意事先指定的实数。
\end{remark}

\subsection{阿贝尔判别法与狄利克雷判别法}
\begin{remark}
In mathematics, summation by parts transforms the summation 
of products of sequences into other summations, often simplifying the 
computation or (especially) estimation of certain types of sums. 
The summation by parts formula is sometimes called Abel's 
lemma or Abel transformation.

Suppose $\left\{f_k\right\} $and $\displaystyle \left\{g_k\right\}$ are two sequences.
Then,$\displaystyle \sum_{k=m}^n f_k(g_{k+1}-g_k) = \left[f_{n}g_{n+1} - f_m g_m\right] 
- \sum_{k=m+1}^n g_{k}(f_{k}- f_{k-1}).$
Using the forward difference operator $\displaystyle \Delta$, it can be stated 
more succinctly as $\displaystyle \sum_{k=m}^n f_k\Delta g_k = \left[f_{n} g_{n+1} 
- f_m g_m\right] - \sum_{k=m}^{n-1} g_{k+1}\Delta f_k,$
Note that summation by parts is an analogue to the integration by parts formula,
$\displaystyle \int f\,dg = f g - \int g\,df.$
\end{remark}

\begin{lemma}{\rm (阿贝尔变换)}
    设$a_i, b_i (i=1,2,\cdots,n)$ 为两组实数,若令
    \[
        B_k = b_1 + b_2 + \cdots + b_k (k = 1,2,\cdots,n)
        \]
    则有:
    \begin{equation}
        \sum_{i=1}^n a_ib_i = (a_1-a_2)B_1 + (a_2-a_3)B_2 + \cdots + (a_{n-1}-a_n)B_{n-1}+a_nB_n
        \label{eq:eq5}
    \end{equation}
\end{lemma}

\begin{corollary}{\rm(阿贝尔引理)}
    若
    \begin{enumerate}[label={\rm(\arabic*)}]
        \item $\displaystyle a_1, a_2, \cdots, a_n$ 是单调数组;
        \item  对于任意正整数$\displaystyle k(1 \le k \le n)$ 有
            $\displaystyle \left|B_k\right| \le M $\rm(这里$\displaystyle B_k = b_1 + b_2 + \cdots + b_k$),
            记$\epsilon  = \max_k\left\{\left|a_k\right|\right\}$
        则有:
            \begin{equation}
                \left|\sum_{k=1}^na_kb_k\right| \le 3 \epsilon M.
                \label{eq:eq6}
            \end{equation}
    \end{enumerate}
\end{corollary}

现在讨论级数
\begin{equation}
    \sum_{k=1}^{\infty} a_kb_k  = a_1b_1 + a_2b_2 + \cdots + a_nb_n  + \cdots
    \label{eq:eq7}
\end{equation}
收敛性的判别法。

\begin{theorem}{\rm (阿贝尔判别法)}
    若$\displaystyle \left\{a_n\right\}$为单调有界数列,且级数
    $\displaystyle \sum b_n$ 收敛,则级数\ref{eq:eq7}收敛。
\end{theorem}

\begin{theorem}{\rm (狄利克雷判别法)}
    若$\displaystyle \left\{a_n\right\}$为单调递减,且$\displaystyle \lim_{n \to \infty} a_n = 0$, 
    $\displaystyle \sum b_n$ 的部分和数列有界,则级数\ref{eq:eq7}收敛。
\end{theorem}

\begin{example}
    设数列$\displaystyle \left\{a_n\right\}$具有性质:
    \[
        a_1 \ge a_2 \ge \cdots \ge a_n \ge \cdots, \lim_{n \to \infty} a_n = 0
        \]
     则级数$\displaystyle \sum a_n \sin nx, \sum a_n \cos nx$对于任意 
     $x \in (0, 2\pi)$都收敛。
     特别的,级数
    \[
        \sum\frac{\sin nx}{n}, \sum\frac{\cos nx}{n}
        \]
    对一切$x \in (0, 2\pi)$都收敛。
\end{example}

\begin{example}
    讨论级数
    \[
        \sum_{n=1}^{\infty} \frac{\sin nx}{n^p}(0 < x < \pi)
        \]
    的敛散性。
\end{example}

\subsection{作业}
\subsubsection{下列级数那些条件收敛,那些绝对收敛,那些发散}
\begin{enumerate}[label={\rm(\arabic*)}]
    \item $\displaystyle \sum \frac{\sin nx}{n!}$
    \item $\displaystyle \sum (-1)^n\frac{n}{n+1}$
    \item $\displaystyle \sum (-1)^n\frac{\ln (n+1)}{n+1}$
    \item $\displaystyle \sum n! \left(\frac{x}{n}\right)^n$
\end{enumerate}
\subsubsection{应用阿贝尔和狄利克雷方法,判断下列级数的敛散}
\begin{enumerate}[label={\rm(\arabic*)}]
    \item $\displaystyle \sum \frac{(-1)^n}{n}\frac{x^n}{x^n+1}, (x > 0)$
    \item $\displaystyle \sum \frac{\sin nx}{n^{\alpha}}, x\in(0,2\pi), 
        \alpha > 0$
    \item $\displaystyle \sum (-1)^n\frac{\cos^2n}{n}$
\end{enumerate}






\end{CJK}
\end{document}

\end{CJK}
\end{document}
