\documentclass[a4paper,12pt]{article} % declaration
\usepackage[utf8]{inputenc}
\usepackage{amsmath}
\usepackage{amsthm,accents}
\usepackage{amsfonts}
\usepackage{color}
\usepackage{graphicx}
\usepackage{tikz}
\usepackage{lmodern,bm}
\usepackage{enumitem}

\newtheorem{definition}{Definition}[section]
\newtheorem{theorem}{Theorem}[section]
\newtheorem{proposition}{Proposition}[section]
\newtheorem{lemma}[theorem]{Lemma}
\newtheorem{corollary}[theorem]{Corollary}
\newtheorem{example}{Examples}
\newtheorem*{remark}{Remark}
\title{Primitive}
\author{Guoning Wu}
\begin{document}
\tableofcontents
\setcounter{tocdepth}{2}
\listoffigures
\listoftables
\maketitle
\theoremstyle{definition}

In differential calculus, as we verified on the examples of previous
section, in addition to knowing how to differentiate functions and 
write relations between their derivatives, it is also very valuable 
to know how to find functions from relations satisfied by their 
derivatives. The simplest such problem, but, as will be seen below, 
a very important one, is the problem of finding a function $F(x)$ knowing 
its derivative $F'(x) = f(x)$.

\section{\rm \textbf{The Primitive and the Indefinite Integral}}

\begin{definition}
    \normalfont 
    A function $F(x)$ is a primitive of a function $f(x)$ on an interval 
    if $F$ is differentiable on the interval and satisfies the equation 
    $F'(x) = f(x)$, or, what is the same, $\mathrm{d}F(x) = f(x)\mathrm{d}x$.
\end{definition}

\begin{example}
    \normalfont
    The function $F(x) = \tan^{-1}x$ is a primitive of the function 
    $f(x) = \frac{1}{1+x^2}$ on the entire real line, since $\left(\tan^{-1}x\right)' = 
    \frac{1}{1+x^2}$.
\end{example}

\begin{example}
    \normalfont 
    The function $F(x) = \cot^{-1}\frac{1}{x}$ is a primitive of $f(x)
    =frac{1}{1+x^2}$ on the set of positive real numbers and on the 
    set of negative real numbers.
\end{example}

\begin{proposition}
    \normalfont 
    If $F_1(x)$ and $F_2(x)$ are two primitives of $f(x)$ on the same interval,
    then the difference $F_1(x) - F_2(x)$ is constant on that interval.
\end{proposition}

Like the operation of taking the differential, the operation of 
finding a primitive has the name "indefinite integration" and 
the mathematical notation 
\begin{equation}
    \int f(x)\mathrm{d}x
\end{equation}
\begin{equation}
    \mathrm{d}\int f(x) \mathrm{d}x = \mathrm{d}F(x) = f(x)\mathrm{d}x
    \label{eq:eq1}
\end{equation}
\begin{equation}
    \int\mathrm{d}F(x) = \int F'(x)\mathrm{d}x = F(x) + C
    \label{eq:eq2}
\end{equation}
Formulas ~\ref{eq:eq1} and ~\ref{eq:eq2} establish a reciprocity 
between operations of differentiation and indefinite integration.

\section{\rm \textbf{The Basic Methods of Finding a Primitive}}
The following rules holds:
\begin{equation}
    \int \alpha u(x) + \beta v(x) \mathrm{d}x = \alpha \int u(x)\mathrm{d}x + 
    \beta \int v(x)\mathrm{d}x
\end{equation}

\begin{equation}
    \int \left(uv\right)'\mathrm{d}x = \int u'v\mathrm{d}x + \int uv'\mathrm{d}x
\end{equation}

\begin{proposition}
    \normalfont
    If $\int f(x)\mathrm{d}x = F(x) + C,$ on an interval,
    $ I_x \text{ and } I_t \to I_x $ is a smooth (continuously 
    differentiable mapping of the interval )$ I_x$, then
    \[
        \int (f\circ\varphi)(t)\varphi'(t)\mathrm{d}t = (F\circ\varphi)(t) + C.
        \]
\end{proposition}

\section{\rm \textbf{Primitive of Basic Elementary Functions}}
\[
    \int \frac{1}{\sqrt{x^2\pm 1}}\mathrm{d}x = \ln \vert x + \sqrt{x^2 \pm 1} \vert + C
    \]
\[
    \int \frac{1}{1 - x^2} \mathrm{d}x = \frac{1}{2}\ln\left\vert \frac{1+x}{1-x}
    \right\vert + C
    \]

\subsection{Linearity of the Indefinite Integral}
\begin{example}
    \[
        \begin{split}
            &\int a_0 + a_1x + \cdots + a_nx^n\mathrm{d}x \\
            & = a_0 \int 1\mathrm{d}x + a_1\int x\mathrm{d}x + \cdots + 
             a_n\int x^n \mathrm{d}x \\
            & = a_0x + \frac{1}{2}a_1x^2 + \cdots + \frac{1}{n+1}a_nx^{n+1}
        \end{split}
            \]
\end{example}

\begin{example}
    \[
        \begin{split}
            & \int \left(x + \frac{1}{\sqrt{x}}\right)^2\mathrm{d}x = 
        \int \left(x^2 + 2\sqrt{x} + \frac{1}{x}\right)\mathrm{d}x \\
            & = \frac{1}{3}x^3 + \frac{4}{3}x^{3/2} + \ln \vert x \vert +C
        \end{split}
        \]
\end{example}

\subsection{Integration by Parts}

\[
    \int u\mathrm{d}v = uv - \int v\mathrm{d}u
    \]
\begin{example}
    \[
        \int \ln x \mathrm{d}x
        \]
\end{example}
\begin{example}
    \[
        \int x^2e^x\mathrm{d}x
        \]
\end{example}

\subsection{Change of Variable in an Indefinite Integral}
\[
    \int \left(f\circ\varphi\right)(t)\varphi'(t)\mathrm{d}t = \int f(\varphi(t))\mathrm{d}\varphi(t) = 
    F(\varphi(t)) + C
    \]

\begin{example}
    \[
        \int \frac{t}{1+t^2}\mathrm{d}t = \frac{1}{2}\int \frac{\mathrm{d}(t^2 + 1)}{1+t^2}
        = \frac{1}{2}\ln (t^2 + 1) + C
        \]
\end{example}

\begin{example}
    \[
        \int \sin 2x \cos 3x \mathrm{d}x
        \]
\end{example}

\begin{example}
    \[
    \int \sin^{-1}x \mathrm{d}x
    \]
\end{example}

\begin{example}
    \[
    \int e^{ax} \cos bx \mathrm{d}x
    \]
\end{example}

We could have arrived at this result by using Euler's formula and the
fact that the primitive of the function $e^{(a+ib)x} = e^{ax}\cos bx + ie^{ax}\sin bx$ 
is 
\[
    \begin{split}
        \frac{1}{a+ib}e^{(a+ib)x}  & = \frac{a-ib}{a^2 + b^2}e^{(a+ib)x} \\
        & = \frac{a\cos bx + b\sin bx}{a^2 + b^2}e^{ax} - 
            i\frac{a\sin bx + b\cos bx}{a^2 + b^2}e^{ax}
    \end{split}
            \]

It should not conflate the phase "finding a primitive" with the impossible 
task of "expressing the primitive of a given elementary function in 
terms of elementary functions.

For example, the \textbf{sine integral} Si $x$ is the primitive
\[
    \normalfont
   Si x =  \int \frac{\sin x}{x} \mathrm{d}x
    \]
 of the function $\frac{\sin x}{x}$ that tends to zero as 
$x \to 0$.

Similarly, the function 
\[
    \normalfont
    Ci x = \int \frac{\cos x}{x} \mathrm{d}x
    \]
specified by the condition$ Ci x \to 0$ as $ x \to \infty$ is not elementary.
The function $Ci x$ is called the \textbf{cosine integral.}

The primitive 
\[
    \normalfont 
   li x =  \int \frac{1}{\ln x}\mathrm{d}x
   \]
   is also not elementary. One of the primitives of this function 
   is denoted as $li x$ and is called \textbf{logarithmic integral.}

\section{Primitive of Rational Functions}
Let us consider the problem of integrating $R(x)\mathrm{d}x$, where 
$R(x) = \frac{P(x)}{Q(x)}$ is a ratio of polynomials.

If we work in the domain of real numbers, then, without going outside 
this domain, we can express every such fraction, as we know from 
algebra as a sum 
\begin{equation}
    \frac{P(x)}{Q(x)} = p(x) + \sum_{j=1}^l \left(\sum_{k=1}^{k_j} \frac{a_{jk}}{(x - x_j)^k}\right)
    + \sum_{j=1}^n\left(\sum_{k=1}^{m_j}\frac{b_{jk}x + c_{jk}}{(x^2 + p_jx + q)^k}\right)
\end{equation}

We have already integrated a polynomial, so that it remains only 
to consider the integration of the forms
\[
    \int\frac{1}{(x-a)^k}\mathrm{d}x  \text{ and } \int \frac{bx + c}{(x^2 + px + q)^k}\mathrm{d}x
    \]
The first of these problems can be solved immediately, since 
\begin{equation}
    \int \frac{1}{(x-a)^k}\mathrm{d}x = \left\{\begin{array}{cl}\frac{1}{-k+1}
        (x-a)^{-k+1}+C & \text{ for } k \ne 1 \\
        \ln \vert x-a \vert +C & \text{ for } k = 1 
    \end{array} \right.
\end{equation}

With the integral 
\[
    \int \frac{bx+c}{(x^2+px+q)^k}\mathrm{d}x
    \]
we proceed as follows. We present the polynomial $x^2+px+q$ as $\left(x+\frac{1}{p}\right)^2
 + \left(q - \frac{1}{4}p^2\right)$, where $q - \frac{1}{4}p^2 > 0$, since the polynomial 
 $x^2+px+q$ has no real roots. Setting $x + \frac{1}{2}p = u$ and $q - \frac{1}{4}p^2
 = a^2$, we obtain 
\[
    \int \frac{bx + c}{(x^2 + px + q)^k}\mathrm{d}x = 
    \int \frac{\alpha u + \beta}{(u^2 + a^2)^k}\mathrm{d}u
    \]
where $\alpha = a, \beta = c - \frac{1}{2}bp.$

Next, 
\[
    \begin{split}
        \int \frac{u}{(u^2 + a^2)^k}\mathrm{d}u & = \frac{1}{2}\int 
        \frac{\mathrm{d}(u^2 + a^2)}{(u^2 + a^2)^k}\\  & = 
        \left\{\begin{array}{cl} \frac{1}{2(1-k)}\left(u^2 + a^2\right)^{-k+1} + C & \text{ for } k \ne 1, \\
            \frac{1}{2}\ln (u^2 + a^2) + C & \text{ for } k = 1 \end{array}\right.
    \end{split}
    \]
and it remains only to study the integral 
\begin{equation}
    I_k = \int \frac{\mathrm{d}u}{(u^2 + a^2)^k}.
    \label{eq:eq3}
\end{equation}

Integrating by parts and making elementary transformations, we have 
\[
    \begin{split}
        I_k  & = \int \frac{\mathrm{d}u}{(u^2 + a^2)^k} = \frac{u}{(u^2 + a^2)^2}
        + 2k \int \frac{u^2\mathrm{d}u}{(u^2 + a^2)^{k+1}} \\ & = 
        \frac{u}{(u^2 + a^2)^k} + 2k \int \frac{(u^2 + a^2) - a^2}{(u^2 + a^2)^{k+1}}
        \mathrm{d}u = \frac{u}{(u^2 + a^2)^k} + 2kI_k - 2ka^2I_{k+1}.
    \end{split}
        \]
from which we obtain the recursion relation
\begin{equation}
    I_{k+1} = \frac{1}{2ka^2}\frac{u}{(u^2 + a^2)^k} + \frac{2k - 1}{2ka^2}I_k
\end{equation}
which makes it possible to lower the exponent $k$ in the integral~\ref{eq:eq3}.
But $I_1$ is easy to compute:
\begin{equation}
    I_1 = \int \frac{\mathrm{d}u}{u^2 + a^2} = \frac{1}{a}\int \frac{\mathrm{d}\left(\frac{u}{a}\right)}
    {1+\left(\frac{u}{a}\right)^2} = \frac{1}{a}\tan^{-1}\frac{u}{a} + C.
\end{equation}

\begin{proposition}
    \normalfont
    The primitive of any rational function $R(x) = \frac{P(x)}{Q(x)}$ can be 
    expressed in terms of rational functions and the transcendental functions 
    $\ln$ and $\tan^{-1}$. The rational part of the primitive, when placed over a 
    common denominator, will have a denominator containing all the factors 
    of the polynomial $Q(x)$ with multiplicities one less that they have in $Q(x)$.
\end{proposition}

\begin{example}
    \normalfont
    Calculate $\displaystyle\int \frac{2x^2 + 5x + 5}{(x^2 - 1)(x + 2)}\mathrm{d}x$
\end{example}

\begin{example}
    \normalfont 
    Calculate $\displaystyle \int \frac{x^7 - 2x^6 + 4x^5 - 5x^4 + 4x^3 - 5x^2 - x}
    {(x - 1)^2(x^2 + 1)^2}\mathrm{d}x$
\end{example}

\section{Primitive of the Form $\displaystyle \int R(\cos x, \sin x)\mathrm{d}x$}
Let $R(u,v)$ be a rational function in $u$ and $v$, that is a quotient of polynomials 
$\displaystyle \frac{P(u,v)}{Q(u,v)}$, which are linear combinations of monomials 
$u^mv^n$, where $m = 1,2,\cdots, n=1,2,\cdots$.

Several methods exist for computing the integral $\displaystyle \int R(\cos x, \sin x)
\mathrm{d}x$, one of which is completely general, although not always the most efficient.
\paragraph{\rm \textbf{a.}} We make the change of variable $t = \tan\frac{x}{2}$. Since 
\[
    \begin{split}
        &\cos x = \frac{1 - \tan^2\frac{x}{2}}{1 + \tan^2\frac{x}{2}},\quad
    \sin x = \frac{2\tan \frac{x}{2}}{1 + \tan^2\frac{x}{2}},\\
        &\mathrm{d}t = \frac{\mathrm{d}x}{2\cos^2\frac{x}{2}},\quad
    \mathrm{d}x = \frac{2\mathrm{d}t}{1 + \tan^2\frac{x}{2}}.
    \end{split}
    \]
as follows that 
\[
    \int R(\cos x, \sin x)\mathrm{d}x = \int R\left(\frac{1-t^2}{1+t^2},\frac{2t}{1+t^2}\right)
    \frac{2}{1+t^2}\mathrm{d}t,
    \]
and the problem has been reduced to integrating a rational function.

However, this way leads to a very cumbersome rational function; for 
that reason one should keep in mind that in many cases there are other 
possibilities for rationalizing the integral.

\paragraph{\rm \textbf{b.}} In the case integral of the form $ \displaystyle 
\int R(\cos^2x, \sin^2x)\mathrm{d}x$ or $\displaystyle \int r(\tan x)\mathrm{d}x$, 
where $r$ is rational function, a convenient substitution is $t = \tan x$, since 
\[
    \begin{split}
        &\cos^2x = \frac{1}{1 + \tan^2x}, \quad \sin^2x = \frac{tan^2x}{1 + \tan^2x}\\
        &\mathrm{d}t = \frac{\mathrm{d}x}{\cos^2x} \Rightarrow \mathrm{d}x = \frac{\mathrm{d}t}
        {1 + t^2}
    \end{split}
    \]
Carrying out this substitution, we obtain respectively 
\[
    \int R(\cos^2x, \sin^2x)\mathrm{d}x = \int R\left(\frac{1}{1+t^2}, \frac{t^2}{1+t^2}\right)
    \frac{\mathrm{d}t}{1+t^2}
    \]
\[
    r(\tan x)\mathrm{d}x = \int r(t)\frac{\mathrm{d}t}{1 + t^2}
    \]
\paragraph{\rm \textbf{c.}} In the case of integrals of the form 
\[
    \int R(\cos x, \sin^2x)\sin x \mathrm{d}x, \quad 
    \int R(\cos^2x, \sin x)\cos x \mathrm{d}x,
    \]
One can move the function $\sin x$ and $\cos x $ into the differential 
and make the substitution $t = \cos x$ or $t = \sin x$ respectively. 
After these substitution, the integrals will have the form 
\[
    -\int R(t, 1-t^2)\mathrm{d}t \text{ or } \int R(1-t^2,t)\mathrm{d}t
    \]

\begin{example}
    \[
        \int \frac{\mathrm{d}x}{3 + \sin x}\mathrm{d}x
    \]
\end{example}

\begin{example}
    \[
        \int \frac{\mathrm{d}x}{(\sin x + \cos x)^2}\mathrm{d}x
    \]
\end{example}

\begin{example}
    \[
        \int \frac{\mathrm{d}x}{2\sin^3x - 3\cos^23x + 1}\mathrm{d}x
    \]
\end{example}

\begin{example}
    \[
        \int \frac{\cos^3x}{\sin^7x}\mathrm{d}x
    \]
\end{example}

\section{Primitive of the Form $\displaystyle \int R(x, y(x))\mathrm{d}x$.}
Let $R(x,y)$ be, as in previous section, a rational function. 
Let us consider some special integrals of the form 
\[
    \int R(x,y(x))\mathrm{d}x
    \]

First of all, it is clear that if one can make a change of variable 
$x = x(t)$ such that both functions $x = x(t)$ and $y = y(t)$ are rational 
functions of $t$, then $x'(t)$ is also a rational function and 
\[
    \int R(x,y(t))\mathrm{d}x = \int R(x(t),y(x(t)))x'(t)\mathrm{d}t
    \]
that is, the problem will have been reduced to integrating a rational 
function.

Consider the following special choices of the function $y = y(x)$.
\paragraph{\rm \textbf{a.}} If $\displaystyle y = \sqrt[n]{\frac{ax+b}
{cx+d}}$, where $n \in \mathbb{N}$, then, setting $\displaystyle t^n = \frac{ax+b}{cx+d}$,
we obtain 
\[
    x = \frac{d\cdot t^n - b}{a - c\cdot t^n}, y = t,
    \]
and the integrand rationalizes.
\begin{example}
    \[
        \int \sqrt[3]{\frac{x-1}{x+1}}\mathrm{d}x
        \]
\end{example}
\paragraph{\rm \textbf{b.}} Let us now consider the case when $y = \sqrt{ax^2 + bx + c}$
, that is, integrals of the form 
\[
    \int R(x,\sqrt{ax^2 + bx + c})\mathrm{d}x
    \]
By completing the square in the trinomial $ax^2 + bx + c$ and making a 
suitable liner substitution, we reduce the general case to one of the 
following three simple cases:
\begin{equation}
    \int R(t,\sqrt{t^2 + 1})\mathrm{d}t, 
    \int R(t,\sqrt{t^2 - 1})\mathrm{d}t, 
    \int R(t,\sqrt{1 - t^2})\mathrm{d}t
    \label{eq:eq4}
\end{equation}

To rationalize these integrals it now suffices to make the following 
substitutions, respectively\footnote{These substitution were proposed
long ago by Euler.}:
\[
    \sqrt{t^2 + 1} = tu + 1, \text{ or } \sqrt{t^2 + 1} = tu - 1, 
    \text{ or }\sqrt{t^2 + 1} = t - u;
    \]
\[
    \sqrt{t^2 - 1} = u(t-1), \text{ or } \sqrt{t^2 - 1} = u(t+1),
    \text{ or } \sqrt{t^2 - 1} = t - u;
    \]
\[
    \sqrt{1 - t^2} = u(1 - t), \text{ or } \sqrt{1 - t^2} = u(1+t),
    \text{ or } \sqrt{t^2 - 1} = tu \pm 1.
    \]

Let us verify, for example, that after the first substitution we will 
have reduced the first integral to the integral of a rational function.

In fact, if $\sqrt{t^2 + 1} = tu + 1$, then $t^2 + 1 = t^2u^2 + 2tu + 1$, 
from which we find 
\[
    t = \frac{2u}{1 - u^2}
    \]
and then 
\[
    \sqrt{t^2 + 1} = \frac{1 + u^2}{1 - u^2}
    \]

The integrals~\ref{eq:eq4} can also be reduced, by means of the 
substitutions $t = \sinh \varphi, t = \cosh \varphi, t = \sin \varphi, \text{ or }
t = \cos \varphi$,  respectively, to the following forms:
\[
    \begin{split}
        &\int R(\sinh \varphi, \cosh \varphi)\cosh \varphi\mathrm{d}\varphi,
    \int R(\cosh \varphi, \sinh \varphi)\sinh \varphi\mathrm{d}\varphi,\\
        & \int R(\sin \varphi, \cos \varphi)\cos \varphi\mathrm{d}\varphi,
    -\int R(\cos \varphi, \sin \varphi)\sin \varphi\mathrm{d}\varphi.
    \end{split}
    \]
\begin{example}
    \[
        \int \frac{\mathrm{d}x}{x + \sqrt{x^2+2x+1}} = 
        \int \frac{\mathrm{d}x}{x + \sqrt{(x+1)^2+1}} = 
        \int \frac{\mathrm{d}t}{t-1+\sqrt{t^2+1}}.
        \]
\end{example}
\begin{proof}
    Setting $\displaystyle \sqrt{t^2+1} = u - t.$
\end{proof}
\paragraph{\rm \textbf{c.Elliptic integrals.}} 
Another important class of integrals consists of those of the form 
\begin{equation}
    \int R(x,\sqrt{P(x)})\mathrm{d}x
    \label{eq:eq5}
\end{equation}
where $P(x)$ is a polynomial of degree of $n>2$. As Abel and Liouville 
showed, such an integral cannot in general be expressed in terms of 
elementary function.

For $n = 3$ and $n = 4$ the integral~\ref{eq:eq5} is called an 
\textbf{elliptic integral,} and for $n > 4$ it is called \textbf{hyperelliptic.}

It can be shown that by elementary substitutions the general 
elliptic integral can be reduced to the following three standard 
forms up to terms expressible in elementary functions:
\begin{equation}
    \int \frac{\mathrm{d}x}{\sqrt{(1-x^2)(1-k^2x^2)}}
\end{equation}
\begin{equation}
    \int \frac{x^2\mathrm{d}x}{\sqrt{(1-x^2)(1-k^2x^2)}}
\end{equation}
\begin{equation}
    \int \frac{\mathrm{d}x}{(1+hx^2)\sqrt{(1-x^2)(1-k^2x^2)}}
\end{equation}
where $h$ and $k$ are parameters, the parameter $k$ lying in the interval 
$]0,1[$ in all three cases.

By the substitution $x = \sin \varphi$ these integrals can be reduced to 
the following canonical integrals and combinations of them:
\begin{equation}
    \int \frac{\mathrm{d}\varphi}{\sqrt{1-k^2\sin^2\varphi}}
    \label{eq:eq6}
\end{equation}
\begin{equation}
    \int \sqrt{1 - k^2\sin^2\varphi}\mathrm{d}\varphi
    \label{eq:eq7}
\end{equation}
\begin{equation}
    \int \frac{\mathrm{d}\varphi}{(1-h\sin^2\varphi)\sqrt{1-k^2\sin^2\varphi}}
    \label{eq:eq8}
\end{equation}

The integrals ~\ref{eq:eq6}, \ref{eq:eq7} and \ref{eq:eq8} are 
called respectively the elliptic integral of \textbf{first kind}, 
\textbf{second kind} and \textbf{third kind}.

The following non-elementary special functions.
\begin{enumerate}
    \normalfont
    \item $\displaystyle Ei(x) = \int \frac{e^x}{x}\mathrm{d}x $,
        the exponential integral.
    \item $ \displaystyle Si(x) = \int \frac{\sin x}{x}\mathrm{d}x$,
        the sine integral.
    \item $\displaystyle Ci(x) = \int \frac{\cos x}{x}\mathrm{d}x$,
        the cosine integral.
    \item $\displaystyle Chi(x) = \int \frac{\cosh x}{x}\mathrm{d}x $,
        the hyperbolic cosine integral.
    \item $\displaystyle Shi(x) = \int \frac{\sinh x}{x}\mathrm{d}x $,
        the hyperbolic sine integral.
    \item $\displaystyle S(x) = \int \sin x^2\mathrm{d}x$ , the Fresnel integral.
    \item $\displaystyle C(x) = \int \cos x^2\mathrm{d}x$ , the Fresnel integral.
    \item $\displaystyle \Phi(x) = \int e^{-x^2}\mathrm{d}x$ , 
        the Euler-Poission integral.
    \item $\displaystyle li(x) = \int \frac{\mathrm{d}x}{\ln x}$ ,
        the logarithmic integral.
\end{enumerate}
\end{document}
