\documentclass[a4paper,12pt]{article}
\usepackage[utf8]{inputenc}
\usepackage{CJKutf8}
\usepackage{multirow}
\usepackage{graphicx}
\usepackage{amsmath}
\usepackage{amssymb}
\usepackage{amsthm}
\usepackage{enumerate}
\usepackage{fancyhdr}
\usepackage{setspace}
\usepackage{enumitem}
\newtheorem{theorem}{定理}
\newtheorem{lemma}{引理}
\newtheorem{definition}{定义}
\newtheorem{example}{例子}
\newtheorem{corollary}{推论}
\newtheorem{remark}{注}
\let\oldref\ref
\renewcommand{\ref}[1]{\rm{(\oldref{#1})}}
\renewcommand{\headrulewidth}{0.4pt}
\renewcommand{\footrulewidth}{0.4pt}

\begin{CJK}{UTF8}{gbsn}
\begin{document}
\title{函数列与函数项级数}
\author{武国宁}
\date{}
\maketitle

\section{一致收敛}
\subsection{函数列及其一致收敛性}
设
\begin{equation}
    f_1, f_2, \cdots, f_n, \cdots
    \label{eq:eq1}
\end{equation}
是一列定义在同一数集$E$上的函数,称为定义在$E$上的函数列,记为:
\[
    \left\{f_n\right\}
\]
设$x_0 \in E$,得到数列:
\begin{equation}
    \left\{ f(x_0) \right\} 
    \label{eq:eq2}
\end{equation}
收敛,则称数列\ref{eq:eq1}在$x_0$点收敛,$x_0$称为数列\ref{eq:eq1}的收敛点。
若数列\ref{eq:eq2}发散,则称函数列\ref{eq:eq1}在$x_0$点发散。若数列\ref{eq:eq1}
在$D \subset E$上的每一点都收敛,则称\ref{eq:eq1}在数集$D$上收敛。这是对于
$\forall x \in D$,都有数列$\left\{f_n(x)\right\}$ 的一个极限值与之对应,由这个对应法则所确定
的$D$上的函数,成为函数列\ref{eq:eq1}的极限函数。若把此极限函数记作$f(x)$,
则有:
\[
    \lim_{n \to \infty} f_n(x) = f(x), \forall x \in D
    \]
\begin{remark}
    函数极限的$\epsilon - N$ 定义是:对于每一个$x \in D$,$\displaystyle \forall x \in D,
    \exists N > 0, \forall n > N, s.t. \left| f_n(x) - f(x) \right| < \epsilon.$
\end{remark}
\begin{remark}
    使得函数$\left\{ f_n(x) \right\}$收敛的全体收敛点的集合,成为函数列$\left\{f_n(x)\right\}$的收敛域。
\end{remark}

\begin{example}
    讨论函数列$\left\{f_n(x) = x^n\right\}$的收敛域与极限函数。
\end{example}

\begin{example}
    讨论函数列$f_n(x) = \frac{\sin nx}{n}$的收敛域与极限函数。
\end{example}

\begin{definition}
    设函数列$f_n(x)$与函数$f(x)$定义在同一数集$D$上,若对于任给的正数
    $\epsilon > 0$,总存在某一个正整数$N > 0$, 使得当$n > N$ 时, 对于一切
    $x \in D$ ,都有:
    \[
        \left|f_n(x) - f(x)\right| < \epsilon,
        \]
    则称函数列$\left\{f_n(x)\right\}$在$D$上\textbf{一致收敛}于$f(x)$,记作:
    \[
        f_n(x) \rightrightarrows f(x), x \in D
        \]
\end{definition}

\begin{theorem}{\rm 函数列一致收敛的柯西准则}
    函数列$f_n(x)$在数集$D$上一致收敛的充分必要条件为:
    对于任给的$\epsilon > 0$ ,总存在正整数$N > 0$, 使得当
    $n, m > N$时,对于一切$x \in D$ 都有:
    \begin{equation}
        \left|f_n(x) - f_m(x) \right| < \epsilon.
    \end{equation}
\end{theorem}

\begin{theorem}
    函数列$\left\{f_n(x)\right\}$在$D$上一致收敛于$f$的充分必要条件是:
    \begin{equation}
        \lim_{n \to \infty} \sup_{x \in D}\left|f_n(x) - f(x)\right| = 0
        \label{eq:eq3}
    \end{equation}
\end{theorem}

\begin{corollary}
    函数列$\left\{f_n(x)\right\}$在$D$上不一致收敛于$f(x)$的充分且必要条件为:
    存在$\left\{x_n\right\} \subset D$,使得$\left\{f_n(x_n) - f(x_n)\right\}$不收敛于0.
\end{corollary}

\begin{example}
    讨论函数列$\left\{f_n(x) = nxe^{-nx^2}\right\}, x \in (0, +\infty)$的一致敛散性。
\end{example}

\begin{definition}
    设函数列$\left\{f_n(x)\right\}$与$f(x)$定义在区间$I$上,多对于任意闭区间$[a,b] \subset I$,
    $\left\{f_n(x)\right\}$在$[a,b]$上一致收敛于$f(x)$,则称$\left\{f_n(x)\right\}$在$I$上内闭一致收敛
    于$f$.
\end{definition}

\subsection{函数项级数及其一致收敛性}
设$\left\{u_n(x)\right\}$是定义在数集$E$上的一个函数列,表达式:
\begin{equation}
    u_1(x) + u_2(x) + \cdots + u_n(x) + \cdots, x \in E
    \label{eq:eq4}
\end{equation}
称为定义在$E$上的函数项级数,记为$\displaystyle \sum_{n=1}^{\infty}u_n(x)$,
称
\begin{equation}
    S_n(x) = \sum_{k=1}^n u_k(x), x \in E, n = 1, 2, \cdots,
    \label{eq:eq5}
\end{equation}
为函数项级数\ref{eq:eq4}的部分和数列。
若$x_0 \in E$,数项级数:
\begin{equation}
    u_1(x_0) + u_2(x_0) + \cdots + u_n(x_0) + \cdots 
    \label{eq:eq6}
\end{equation}
收敛,则称$x_0$为函数项级数\ref{eq:eq4}的收敛点。所有的收敛点形成收敛域。
在收敛域上,级数\ref{eq:eq4}对应和函数,并写作:
\[
    u_1(x) + u_2(x) + \cdots + u_n(x) + \cdots = S(x), x \in D
    \]
\begin{example}
    讨论几何级数$\displaystyle \sum_{n=1}^{\infty}x^n$的收敛域。
\end{example}

\begin{definition}
    设$\left\{S_n(x)\right\}$是函数项级数$\displaystyle \sum_{n=1}^{\infty}u_n(x)$的部分和函数列。
    若$\left\{S_n(x)\right\}$在数集$D$上一致收敛于$S(x)$, 则称级数$\sum u_n(x)$在$D$上一致收敛于
    $S(x)$,若$\displaystyle \sum u_n(x)$在任意闭区间$[a,b] \subset I$上一致收敛,则称
    $\displaystyle \sum u_n(x)$在$I$上内闭一致收敛。
\end{definition}
\begin{theorem}
    函数项级数$\displaystyle \sum u_n(x)$在数集$D$上一致收敛的充分必要条件为:
    对于任意给定的正数 $\epsilon$,总存在某个正整数$N$, 使得当$n > N$时,对于一切
    $x \in D$ 和一切正整数$q$,都有:
    \[
        \left|S_{n+p}(x) - S_n(x)\right| < \epsilon
        \]
\end{theorem}

\begin{corollary}
    函数项级数$\displaystyle \sum u_n(x)$ 在数集$D$上一致收敛的必要条件为
    函数列$\displaystyle \left\{u_n(x)\right\}$在$D$上一致收敛于零。
\end{corollary}

设函数项级数$\displaystyle \sum u_n(x)$在$D$上的和函数为$S(x)$, 则称
\[
    R_n(x) = S(x) - S_n(x)
    \]
为函数项级数$\displaystyle \sum u_n(x)$的\textbf{余项}。

\begin{theorem}
    函数项级数$\displaystyle \sum u_n(x)$在数集$D$上一致收敛于$S(x)$的充分必要
    条件为:
    \[
        \lim_{n \to \infty}\sup_{x \in D} \left|R_n(x)\right| = 
        \lim_{n \to \infty}\sup_{x \in D} \left|S(x) - S_n(x)\right| = 0
        \]
\end{theorem}

\begin{example}
    讨论函数项级数$\displaystyle \sum_{n=0}^{\infty}x^n$的一致收敛性。
\end{example}

\subsection{函数项级数一致收敛性的判别方法}
\begin{theorem}{\rm (威尔斯特拉斯判别法)}
    设函数项级数$\displaystyle \sum u_n(x)$定义在数集$D$上,$\displaystyle \sum M_n$
    为收敛的正项级数,若对于一切$x \in D$,有:
    \begin{equation}
        \left|u_n(x)\right| \le M_n, n = 1, 2, \cdots,
        \label{eq:eq7}
    \end{equation}
    则函数项级数$\displaystyle \sum u_n(x)$在$D$上一致收敛。
\end{theorem}

\begin{example}
    讨论函数项级数
    \[
        \sum \frac{\sin nx}{n^2}, \sum \frac{\cos nx}{n^2}
        \]
    在$\displaystyle \left(-\infty, +\infty\right)$的一致收敛性。
\end{example}

\begin{remark}
    上述级数$\displaystyle \sum M_n$称为函数项级数$\displaystyle \sum u_n(x)$的优级数。
    上述判别方法成为\textbf{M}判别法或优级数判别法。
\end{remark}

下面讨论形如
\begin{equation}
    \sum u_n(x)v_n(x)
    \label{eq:eq8}
\end{equation}
\begin{theorem}{\rm(阿贝尔判别法)}设
    \begin{enumerate}[label={\rm(\arabic*)}]
        \item $\displaystyle \sum u_n(x)$ 在区间$I$上一致收敛;
        \item 对于每一个$x \in I, \left\{v_n(x)\right\}$是单调的;
        \item $\displaystyle \left\{v_n(x)\right\}$在$I$上一致有界,即存在正数$M$,
            对于一切$x \in I$ 和正整数$n$,有
            \[
                \left|v_n(x)\right| \le M,
                \]
            则级数\ref{eq:eq8}一致收敛。
    \end{enumerate}
\end{theorem}

\begin{theorem}{\rm(狄利克雷判别法)}设
    \begin{enumerate}[label={\rm(\arabic*)}]
        \item $\displaystyle \sum u_n(x)$的部分和函数列
            \[
                U_n(x) = \sum_{n=1}^{\infty} u_k(x) (n = 1, 2, \cdots,)
                \]
            在$I$上一致有界;
        \item 对于每一个$x \in I, \left\{v_n(x)\right\}$是单调的;
        \item 在$I$上$\displaystyle v_n(x) \rightrightarrows 0(n \to \infty)$
            则级数\ref{eq:eq8}一致收敛。
    \end{enumerate}
\end{theorem}

\begin{example}
    讨论函数项级数
    \[
        \sum \frac{(-1)^n(x+n)^n}{n^{n+1}}
        \]
    在$[0,1]$上的一致敛散性。
\end{example}

\begin{example}
    若数列$\displaystyle \left\{a_n\right\}$单调且收敛于零,则级数
    \[
        \sum a_n \cos nx
        \]
    在$[\alpha, 2\pi-\alpha](0 < \alpha < \pi)$上的一致敛散性。
\end{example}

\subsection{作业}
\subsubsection{讨论下列函数列在所示区间上是否一致收敛或内闭一致收敛,说明理由}
\begin{enumerate}[label={\rm(\arabic*)}]
    \item $\displaystyle f_n(x) = \frac{x}{1 + n^2x^2}, n = 1,2, \cdots, D \in (-\infty, +\infty)$
    \item $\displaystyle f_n(x) = \left\{\begin{array}{cl} 
             -(n+1)x+1, & 0 \le x \le \frac{1}{n+1}, \\
             0        , & \frac{1}{n+1} < x <1. 
            \end{array} \right.$
          $\displaystyle n = 1,2, \cdots$
      \item $\displaystyle f_n(x) = \sin \frac{x}{n}, n = 1, 2, \cdots,
          D \in \left(-\infty, +\infty\right)$
\end{enumerate}

\subsubsection{判别下列函数项级数在所示区间上的一致收敛性}
\begin{enumerate}[label={\rm(\arabic*)}]
    \item $\displaystyle \sum \frac{x^n}{n+1}, x \in [-r, r]$
    \item $\displaystyle \sum \frac{(-1)^{n-1}x^2}{(1+x^2)^n}, x \in 
        \left(-\infty, +\infty\right)$
    \item $\displaystyle \sum \frac{x^n}{n^2}, x \in [0,1]$
    \item $\displaystyle \sum \frac{x^2}{(1+x^2)^{n-1}}, x \in 
        \left(-\infty, +\infty\right)$
\end{enumerate}
\subsubsection{证明题}
证明:$\displaystyle f_n(x)$在区间$I$上内闭一致收敛于$f$的充分且必要条件是:
对于任意$x_0 \in I$,存在$x_0$的一个邻域$\displaystyle U(x_0)$,使得
$\displaystyle \left\{f_n(x)\right\}$在$\displaystyle U(x_0) \cap I$上一致收敛于$f$.

\section{一致收敛函数列于函数项级数的性质}
\begin{theorem}
    设函数列$\displaystyle f_n$在$(a,x_0) \cup (x_0, b)$上一致收敛于$f(x)$,且
    对每一个$ \displaystyle n, \lim_{x \to x_0}f_n(x) = a_n$,
    则$\displaystyle \lim_{n \to \infty} a_n$
    和$\lim_{x \to x_0}f(x)$均存在且相等。
\end{theorem}

\begin{remark}
    上述定理说明:
    \begin{equation}
        \lim_{n \to \infty}\lim_{x \to x_0}f_n(x) = \lim_{x \to x_0}\lim_{n \to \infty}f_n(x)
        \label{eq:eq9}
    \end{equation}
\end{remark}
\begin{remark}
    类似的,若函数$\displaystyle f_n(x)$在$(a,b)$上一致收敛且
    $\displaystyle \lim_{x \to a^+}f_n(x)$存在,可得到:
    \[
        \lim_{n \to \infty}\lim_{x \to a^+}f_n(x) = \lim_{x \to a^+}\lim_{n \to \infty}f_n(x)
        \]
\end{remark}
\begin{remark}
    类似的,若函数$\displaystyle f_n(x)$在$(a,b)$上一致收敛且
    $\displaystyle \lim_{x \to b^-}f_n(x)$存在,可得到:
    \[
        \lim_{n \to \infty}\lim_{x \to b^-}f_n(x) = \lim_{x \to b^-}\lim_{n \to \infty}f_n(x)
        \]
\end{remark}

\begin{theorem}{\rm (连续性)}
    若函数列$\displaystyle \left\{f_n(x)\right\}$在区间$I$上一致连续,且每一项都连续,
    则其极限函数$f$在$I$上连续。
\end{theorem}

\begin{example}
    例如函数列$\displaystyle \left\{x^n\right\}$各项在$(-1, 1]$上连续,但是极限函数为:
     \[
         f(x) = \left\{\begin{array}{ll} 0, & -1 < x < 1, \\
                                   1, & x = 1 
                 \end{array}\right.
                 \]
             说明该函数列在$(-1,1])$上不一致收敛。
\end{example}

\begin{corollary}
    若连续函数列$\left\{f_n\right\}$在区间$I$上内闭一致收敛于$f$,则$f$在$I$上连续。
\end{corollary}

\begin{example}
    例如函数列$\displaystyle \left\{x^n\right\}$各项在$(-1, 1)$上连续,内闭一致收敛
    于$f$,则 $f$ 在$I$上连续。
\end{example}

\begin{theorem}{\rm (可积性)}
    若函数列$\left\{f_n\right\}$在$[a,b]$上一致收敛,且每一项都连续,则有:
    \begin{equation}
        \int_a^b\lim_{n \to \infty}f_n(x)\,\rm{d}x = 
        \lim_{n \to \infty}\int_a^bf_n(x)\,\rm{d}x
        \label{eq:eq10}
    \end{equation}
\end{theorem}


\begin{example}
    讨论函数
    \[
        f_n(x) = \left\{\begin{array}{ll} 2n\alpha_nx,& 0 \le x < \frac{1}{2n} \\
                        2\alpha_n - 2n\alpha_nx, & \frac{1}{2n} \le x < \frac{1}{n} \\
                        0,           & \frac{1}{n} \le x \le 1
        \end{array}\right.
    \]
    的一致收敛及其极限函数的可积性。
\end{example}

\end{CJK}
\end{document}
