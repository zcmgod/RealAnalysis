\documentclass[a4paper,12pt]{article}
\usepackage[utf8]{inputenc}
\usepackage{CJKutf8}
\usepackage{multirow}
\usepackage{graphicx}
\usepackage{amsmath}
\usepackage{amssymb}
\usepackage{amsthm}
\usepackage{enumerate}
\usepackage{fancyhdr}
\usepackage{setspace}
\usepackage{enumitem}
\usepackage{indentfirst}
\addtolength{\parskip}{3pt}
\linespread{1.3} %一倍半行距
\begin{document}
\begin{CJK*}{UTF8}{bsmi}
\newtheorem{theorem}{定理}
\newtheorem{lemma}{引理}
\newtheorem{definition}{定義}
\newtheorem{example}{例子}
\newtheorem{corollary}{推論}
\newtheorem{remark}{注}
\let\oldref\ref
\renewcommand{\ref}[1]{\rm{(\oldref{#1})}}
\renewcommand{\headrulewidth}{0.4pt}
\renewcommand{\footrulewidth}{0.4pt}

\title{Homework}
\author{武國寧}
\date{}
\maketitle
\section{在指定的區間上把下列函數展開稱為傅立葉級數}
    \begin{enumerate}[label={\rm(\arabic*)}]
        \item $\displaystyle f(x) = x, (i)(-\pi, \pi), (ii) (0, 2\pi)$
        \item $\displaystyle f(x) = x^2, (i)(-\pi, \pi), (ii) (0, 2\pi)$
    \end{enumerate}

\section{把函數$f(x)$展開成傅立葉級數}
    \[
        f(x) = \left\{ \begin{array}{cc} -\frac{\pi}{4}, & -\pi < x < 0 \\
                                          \frac{\pi}{4}, &  0 \le x < \pi
                        \end{array}
               \right.
    \]
    並推出下列結果:
    \begin{enumerate}[label={\rm(\arabic*)}]
        \item $\displaystyle \frac{\pi}{4} = 1 - \frac{1}{3} 
               + \frac{1}{5} - \frac{1}{7} + \cdots$
        \item $\displaystyle \frac{\pi}{3} = 1 + \frac{1}{5} 
            - \frac{1}{7} - \frac{1}{11} + \frac{1}{13} 
            +\frac{1}{17} + \cdots$
        \item $\displaystyle \frac{\sqrt{3}}{6}\pi = 1 - \frac{1}{5} 
            + \frac{1}{7} - \frac{1}{11} + \frac{1}{13} - \frac{1}{17}
            +\cdots$
    \end{enumerate}
    
\section{求下列函數$f(x)$的傅立葉級數展式}
    \begin{enumerate}[label={\rm(\arabic*)}]
        \item $\displaystyle f(x) = \frac{\pi - x}{2}, x \in (0, 2\pi)$
        \item $\displaystyle f(x) = \sqrt{1 - \cos x}, x \in (-\pi, \pi)$
        \item $\displaystyle f(x) = ax^2 + bx + c, (i)x \in (-\pi, \pi), 
                (ii) x \in (0, 2\pi)$
        \item $\displaystyle f(x) = ch x, x \in (-\pi, \pi)$
    \end{enumerate}

\end{CJK*}
\end{document}


\end{CJK*}
\end{document}
