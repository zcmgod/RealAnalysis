\documentclass[a4paper,12pt]{article}
\usepackage{CJKutf8}
\usepackage{multirow}
\usepackage{graphicx}
\usepackage{amsmath}
\usepackage{amsfonts}
\usepackage{amssymb}
\usepackage{amsthm}
\usepackage{enumerate}
\usepackage{fancyhdr}
\usepackage{setspace}
\usepackage{bm}
%\pagestyle{fancy}
%\lhead{}
%\rhead{\bfseries Integral on lines and surfaces}
%\lfoot{Calculus}
%\cfoot{China University of Petroleum-Beijing}
%\rfoot{\thepage}
%\renewcommand{\headrulewidth}{0.4pt}
%\renewcommand{\footrulewidth}{0.4pt}
%%%% 段落首行缩进两个字 %%%%
\makeatletter
\let\@afterindentfalse\@afterindenttrue
\@afterindenttrue
\makeatother
\setlength{\parindent}{2em}  %中文缩进两个汉字位

\newcounter{solution}
\newcommand\TheSolution{%
  \textbf{Solution:}\\%
}

%%%% 下面的命令重定义页面边距,使其符合中文刊物习惯 %%%%
\addtolength{\topmargin}{-54pt}
\setlength{\oddsidemargin}{0.63cm}  % 3.17cm - 1 inch
\setlength{\evensidemargin}{\oddsidemargin}
\setlength{\textwidth}{14.66cm}
\setlength{\textheight}{24.00cm}    % 24.62

%%%% 下面的命令设置行间距与段落间距 %%%%
\linespread{1.4}
\setlength{\parskip}{1ex}
\setlength{\parskip}{0.5\baselineskip}

%%%% 定理类环境的定义 %%%%
\newtheorem{example}{Example}             % 整体编号
\newtheorem{algorithm}{算法}
\newtheorem{theorem}{Theorem}[section]  % 按 section 编号
\newtheorem{definition}{Definition}
\newtheorem{axiom}{公理}
\newtheorem{property}{性质}
\newtheorem{proposition}{Proposition}
\newtheorem{lemma}{引理}
\newtheorem{corollary}{Corollary}
\newtheorem{remark}{注解}
\newtheorem{condition}{条件}
\newtheorem{conclusion}{结论}
\newtheorem{assumption}{假设}

%%%% 重定义 %%%%
\renewcommand{\contentsname}{目录}  % 将Contents改为目录
\renewcommand{\abstractname}{摘要}  % 将Abstract改为摘要
\renewcommand{\refname}{参考文献}   % 将References改为参考文献
\renewcommand{\indexname}{索引}
%\renewcommand{\figurenamGe}{图}
\renewcommand{\tablename}{表}
\renewcommand{\appendixname}{附录}
\renewcommand{\algorithm}{算法}
\begin{document}
\begin{CJK}{UTF8}{gbsn}
\title{Fourier Series and the Fourier Transform}
\author{Guoning Wu\footnote{Email: wuguoning@163.com}\\[2ex]
 China University of Petroleum-Beijing\\[2ex]}
\date{2017.9}
\maketitle
\section{Basic General Concept Connected with Fourier Series}

\subsection{Orthogonal Systems of Functions}
    \paragraph{\rm \textbf{a. Expansion of a vector in a vector space.}}
    During this course of analysis we have mentioned several times that 
    certain classes of functions form vector spaces in relation to the 
    standard arithmetic operations. Such, for example, are the basic 
    classes of analysis, which consist of smooth, continuous, or integrable 
    real, complex, or vector-valued functions on a domain $X \subset \mathbb{R}^n$.

    In analysis, as a rule, it is necessary to consider "infinite linear
    combinations" - series of functions of the form
    \[
        f = \sum_{k=1}^{\infty}\alpha_k f_k
    \]
    
    The definition of the sum of the series requires that some topology (
    in particular, a metric) be defined in the vector space in question, 
    making it possible to judge whether the difference $f - S_n$ tends to 
    zero or not, where $S_n = \sum_{k=1}^{n}\alpha_kf_k.$

    The main device used in classical analysis to introduce a metric on
    a vector space is to define some norm of a vector or inner product
    of vectors in that space. We are now going to consider only spaces 
    endowed with an inner product (which, as before, we shall denote 
    $\langle, \rangle$)

    \begin{definition}
        The vectors $\bm{x}$ and $\bm{y}$ in a vector space endowed with an 
        inner product $\langle, \rangle$ are orthogonal (with respect to 
        that inner product) if $\langle x, y \rangle = 0.$
    \end{definition}

    \begin{definition}
        The system of vectors $\left\{x_k: k \in K \right\}$ is orthogonal if the
        vectors in it corresponding to different values of the index $k$
        are pairwise orthogonal.
    \end{definition}

    \begin{definition}
        The system of vectors $\left\{e_k: k \in K \right\}$ is orthonormalized (or orthonormal)
        if $\langle e_i, e_j \rangle = \delta_{ij}$ for every pair of indices $i,j \in K$,
        where $\delta_{i,j}$ is the  Kronecker  symbol, that is  
        \[
        \delta_{i,j}  =  \left\{\begin{array}{cr}  & 1, if i=j \\ & 0, if i \ne j 
        \end{array} \right.
        \].
    \end{definition}

    \begin{definition}
        A finite system of vectors $x_1, x_2, \cdots , x_n$ is linearly independent 
        if the equality $\alpha_1x_1 + \alpha_2x_2 + \cdots + \alpha_nx_n = 0$ is possible 
        only when $\alpha_1 = \alpha_2 = \cdots = \alpha_n = 0$.
    \end{definition}


\end{CJK}
\end{document}
