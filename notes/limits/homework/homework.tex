\documentclass[a4paper,12pt]{article}
\usepackage[utf8]{inputenc}
\usepackage{CJKutf8}
\usepackage{multirow}
\usepackage{graphicx}
\usepackage{amsmath}
\usepackage{amssymb}
\usepackage{amsthm}
\usepackage{enumerate}
\usepackage{fancyhdr}
\usepackage{setspace}
\usepackage{enumitem}
\newtheorem{theorem}{定理}
\newtheorem{lemma}{引理}
\newtheorem{definition}{定义}
\newtheorem{example}{例子}
\newtheorem{corollary}{推论}
\newtheorem{remark}{注}
\let\oldref\ref
\renewcommand{\ref}[1]{\rm{(\oldref{#1})}}
\renewcommand{\headrulewidth}{0.4pt}
\renewcommand{\footrulewidth}{0.4pt}

\begin{CJK*}{UTF8}{bsmi}
\begin{document}
\title{作業}
\author{武國寧}
\date{}
\maketitle
 
\section{解答題}
設$\displaystyle a_n = \frac{1+(-1)^n}{n}, n = 1, 2, \cdots, a = 0.$
\begin{enumerate}[label={\rm(\arabic*)}]
    \item 對下列 $\epsilon $分別求出極限定義中的$N:$
        \[
            \epsilon_1 = 0.1, \epsilon_2 = 0.01, \epsilon_3 = 0.001
        \]
    \item 對$\epsilon_1, \epsilon_2, \epsilon_3$,可找到響應的$N$,這是否說明$a_n$趨於0?應該怎樣做才對?
    \item 對於任意給定的$\epsilon$是否可以找到一個$N$?
\end{enumerate}

\section{證明題}
按$\epsilon-N$定義證明:
\begin{enumerate}[label={\rm(\arabic*)}]
    \item $\displaystyle \lim_{n \to \infty} \frac{n}{n+1} = 1$
    \item $\displaystyle \lim_{n \to \infty} \frac{3n^2+n}{2n^2-1} = \frac{3}{2}$
    \item $\displaystyle \lim_{n \to \infty} \frac{n!}{n^n} = 0$
    \item $\displaystyle \lim_{n \to \infty} \frac{n}{a^n} = 0 (a>1)$
    \item $\displaystyle \lim_{n \to \infty} \sqrt[n]{10} = 1$
\end{enumerate}

\section{證明題}
證明:若$\displaystyle \lim_{n \to \infty} a_n = a$,則對於任意的$k$,
有$\displaystyle \lim_{n \to \infty} a_{n+k} = a$

\section{解答題}
下面那些數列是有界數列、無界數列以及無窮大量:
\begin{enumerate}[label={\rm(\arabic*)}]
    \item $\displaystyle \left\{\left[1+(-1)^n\right]\sqrt{n}\right\}$
    \item $\displaystyle \left\{\sin n\right\}$
    \item $\displaystyle \left\{\frac{n^2}{n-\sqrt{5}}\right\}$
    \item $\displaystyle \left\{2^{(-1)^{n}n}\right\}$
\end{enumerate}



\end{CJK*}
\end{document}

