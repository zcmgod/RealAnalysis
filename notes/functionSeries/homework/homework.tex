\documentclass[a4paper,12pt]{article}
\usepackage[utf8]{inputenc}
\usepackage{CJKutf8}
\usepackage{multirow}
\usepackage{graphicx}
\usepackage{amsmath}
\usepackage{amssymb}
\usepackage{amsthm}
\usepackage{enumerate}
\usepackage{fancyhdr}
\usepackage{setspace}
\usepackage{enumitem}
\newtheorem{theorem}{定理}
\newtheorem{lemma}{引理}
\newtheorem{definition}{定义}
\newtheorem{example}{例子}
\newtheorem{corollary}{推论}
\newtheorem{remark}{注}
\let\oldref\ref
\renewcommand{\ref}[1]{\rm{(\oldref{#1})}}
\renewcommand{\headrulewidth}{0.4pt}
\renewcommand{\footrulewidth}{0.4pt}

\begin{CJK}{UTF8}{gbsn}
\begin{document}
\title{Homework}
\author{武国宁}
\date{}
\maketitle

\section{讨论下列函数列在所示区间上是否一致收敛或内闭一致收敛,说明理由}
\begin{enumerate}[label={\rm(\arabic*)}]
    \item $\displaystyle f_n(x) = \frac{x}{1 + n^2x^2}, n = 1,2, \cdots, D \in (-\infty, +\infty)$
    \item $\displaystyle f_n(x) = \left\{\begin{array}{cl} 
             -(n+1)x+1, & 0 \le x \le \frac{1}{n+1}, \\
             0        , & \frac{1}{n+1} < x <1. 
            \end{array} \right.$
          $\displaystyle n = 1,2, \cdots$
      \item $\displaystyle f_n(x) = \sin \frac{x}{n}, n = 1, 2, \cdots,
          D \in \left(-\infty, +\infty\right)$
\end{enumerate}

\section{判别下列函数项级数在所示区间上的一致收敛性}
\begin{enumerate}[label={\rm(\arabic*)}]
    \item $\displaystyle \sum \frac{x^n}{n+1}, x \in [-r, r]$
    \item $\displaystyle \sum \frac{(-1)^{n-1}x^2}{(1+x^2)^n}, x \in 
        \left(-\infty, +\infty\right)$
    \item $\displaystyle \sum \frac{x^n}{n^2}, x \in [0,1]$
    \item $\displaystyle \sum \frac{x^2}{(1+x^2)^{n-1}}, x \in 
        \left(-\infty, +\infty\right)$
\end{enumerate}
\section{证明题}
证明:$\displaystyle f_n(x)$在区间$I$上内闭一致收敛于$f$的充分且必要条件是:
对于任意$x_0 \in I$,存在$x_0$的一个邻域$\displaystyle U(x_0)$,使得
$\displaystyle \left\{f_n(x)\right\}$在$\displaystyle U(x_0) \cap I$上一致收敛于$f$.

\section{讨论下列各函数列在所定义的区间上:}
(a)$\displaystyle \left\{f_n(x)\right\}$ 与 $\left\{f'_n(x)\right\}$
的一致收敛性;

(b)$\displaystyle \left\{f_n(x)\right\}$是否有连续,可积和可导定理的条件
与结论。

\begin{enumerate}[label={\arabic*}]
    \item $\displaystyle f_n(x) = \frac{2x+n}{x+n}, x \in [0,b]$
    \item $\displaystyle f_n(x) = x - \frac{x^n}{n}, x \in [0,1]$
    \item $ \displaystyle f_n(x) = nxe^{-nx^2}, x \in [0,1]$ 
\end{enumerate}

\subsection{\rm (计算题)}
设$\displaystyle S(x) = \sum_{n=1}^{\infty} \frac{x^{n-1}}{n^2}, x \in [-1, ]$
计算
\[
    \int_0^xS(t)\, {\rm{d}}t
    \]
\end{CJK}
\end{document}
