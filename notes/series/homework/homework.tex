\documentclass[a4paper,12pt]{article}
\usepackage[utf8]{inputenc}
\usepackage{CJKutf8}
\usepackage{multirow}
\usepackage{graphicx}
\usepackage{amsmath}
\usepackage{amsthm}
\usepackage{enumerate}
\usepackage{fancyhdr}
\usepackage{setspace}
\usepackage{enumitem}
\newtheorem{theorem}{定理}
\newtheorem{lemma}{引理}
\newtheorem{definition}{定义}
\newtheorem{example}{例子}
\newtheorem{corollary}{推论}
\newtheorem{remark}{注}
\let\oldref\ref
\renewcommand{\ref}[1]{\rm{(\oldref{#1})}}
\renewcommand{\headrulewidth}{0.4pt}
\renewcommand{\footrulewidth}{0.4pt}

\begin{CJK}{UTF8}{gbsn}
\begin{document}
\title{作业}
\author{武国宁}
\date{}
\maketitle

\section{证明下列级数收敛,并求其和}
\begin{enumerate}[label={\rm(\arabic*)}]
    \item $\displaystyle \left(\frac{1}{2} + \frac{1}{3}\right) + 
        \left(\frac{1}{2^2} + \frac{1}{3^2}\right) + \cdots + 
        \left(\frac{1}{2^n} + \frac{1}{3^n}\right)$
    \item$ \displaystyle \sum_{n=1}^{\infty} \frac{1}{n(n+1)(n+2)}$ 
    \item$ \displaystyle \sum_{n=1}^{\infty} \left(\sqrt{n+2} - 2\sqrt{n+1}
        + \sqrt{n} \right)$ 
\end{enumerate}

\section{证明题}
证明:若数列$\left\{a_n\right\}$收敛于$a$,则级数$\displaystyle \sum_{n=1}^{\infty} \left( a_n - 
a_{n+1}\right) = a_1 - a$.  
\section{证明题}
证明:若数列$\left\{b_n\right\}$ 有$ \displaystyle \lim_{n \to \infty} b_n = \infty$, 则:
\begin{enumerate}[label={\rm(\arabic*)}]
    \item 级数$\displaystyle \sum_{n=1}^{\infty} \left(b_{n+1} - b_n\right)$发散;
    \item 当$\displaystyle b_n \ne 0$时,级数$\displaystyle \sum_{n = 1} ^{\infty} 
        \left( \frac{1}{b_n} - \frac{1}{b_{n+1}}\right) = \frac{1}{b_1}$.
\end{enumerate}
\section{利用上述结果求下列级数的和}
\begin{enumerate}[label={\rm(\arabic*)}]
    \item $\displaystyle \sum_{n=1}^{\infty} \frac{1}{(a+n-1)(a+n)}$
    \item $\displaystyle \sum_{n=1}^{\infty} (-1)^{n+1} \frac{2n+1}{n(n+1)}$
\end{enumerate}

\section{应用柯西收敛原理证明下列级数的敛散性}
\begin{enumerate}[label={\rm(\arabic*)}]
    \item $\displaystyle \sum_{n=1}^{\infty} \frac{\sin 2^n}{2^n}$
    \item $\displaystyle \sum_{n=1}^{\infty} \frac{1}{\sqrt{n+n^2}}$
\end{enumerate}
 
\section{判别下列级数的敛散性}
\begin{enumerate}[label={\rm(\arabic*)}]
    \item $\displaystyle \sum \frac{1}{n^2 + a^2}$
    \item $\displaystyle \sum 2^n\frac{\pi}{3^n}$
    \item $\displaystyle \sum \frac{\pi}{n\sqrt[n]n}$
    \item $\displaystyle \sum \frac{(n+1)!}{10^n}$
    \item $\displaystyle \sum \frac{n^2}{2^n}$
\end{enumerate}

\section{采用积分判别法判别下列级数的敛散性}
\begin{enumerate}[label={\rm(\arabic*)}]
    \item $\displaystyle \sum \frac{1}{n^2 + 1}$
    \item $\displaystyle \sum \frac{n}{n^2 + 1}$
\end{enumerate}

\section{证明题}
设$\displaystyle a_n \ge 0, n = 1,2, \cdots$ 且$\displaystyle \left\{na_n\right\}$
有界,证明$\displaystyle a_n^2$收敛。 

\section{下列级数那些条件收敛,那些绝对收敛,那些发散}
\begin{enumerate}[label={\rm(\arabic*)}]
    \item $\displaystyle \sum \frac{\sin nx}{n!}$
    \item $\displaystyle \sum (-1)^n\frac{n}{n+1}$
    \item $\displaystyle \sum (-1)^n\frac{\ln (n+1)}{n+1}$
    \item $\displaystyle \sum n! \left(\frac{x}{n}\right)^n$
\end{enumerate}
\section{应用阿贝尔和狄利克雷方法,判断下列级数的敛散}
\begin{enumerate}[label={\rm(\arabic*)}]
    \item $\displaystyle \sum \frac{(-1)^n}{n}\frac{x^n}{x^n+1}, (x > 0)$
    \item $\displaystyle \sum \frac{\sin nx}{n^{\alpha}}, x\in(0,2\pi), 
        \alpha > 0$
    \item $\displaystyle \sum (-1)^n\frac{\cos^2n}{n}$
\end{enumerate}

\end{CJK}
\end{document}

\end{CJK}
\end{document}
