\documentclass[a4paper,12pt]{article}
\usepackage[utf8]{inputenc}
\usepackage{CJKutf8}
\usepackage{multirow}
\usepackage{graphicx}
\usepackage{amsmath}
\usepackage{amssymb}
\usepackage{amsthm}
\usepackage{enumerate}
\usepackage{fancyhdr}
\usepackage{setspace}
\usepackage{enumitem}
\usepackage{indentfirst}
\addtolength{\parskip}{3pt}
\linespread{1.3} %一倍半行距
\begin{document}
\begin{CJK*}{UTF8}{bsmi}
\newtheorem{theorem}{定理}
\newtheorem{lemma}{引理}
\newtheorem{definition}{定義}
\newtheorem{example}{例子}
\newtheorem{corollary}{推論}
\newtheorem{remark}{注}
\let\oldref\ref
\renewcommand{\ref}[1]{\rm{(\oldref{#1})}}
\renewcommand{\headrulewidth}{0.4pt}
\renewcommand{\footrulewidth}{0.4pt}

\title{冪級數}
\author{武國寧}
\date{}
\maketitle
\section{冪級數}
    \indent 本章將討論由冪函數序列$\left\{a_n(x - x_0)^n\right\}$所產生的函數級數:
    \begin{equation}
        \sum_{n=0}^{\infty}a_n(x - x_0)^n
        \label{eq:eq1}
    \end{equation}
它稱為冪級數,是一種最簡單的函數項級數。它可以看成是多項式的延伸。

    \indent 下面討論$x_0 = 0$,則\ref{eq:eq1}為:
    \begin{equation}
        \sum_{n=0}^{\infty}a_n x^n
        \label{eq:eq2}
    \end{equation}
\subsection{冪級數的收斂區間}
    \indent 首先討論冪級數\ref{eq:eq2}的收斂問題。
    \begin{theorem}{\rm(Abel 定理)}
        若冪級數\ref{eq:eq2}在$x = \bar{x} \ne 0$處收斂,則對於滿足
        不等式:$\vert x \vert < \vert \bar{x} \vert$的任何$x$,
        冪級數\ref{eq:eq2}收斂且絕對收斂;若冪級數\ref{eq:eq2}
        在$x = \bar{x} \ne 0$處發散,則對於滿足
        不等式:$\vert x \vert > \vert \bar{x} \vert$的任何$x$,
        冪級數\ref{eq:eq2}發散。
    \end{theorem}
\begin{remark}
    \indent 冪級數\ref{eq:eq2}的收斂域是以原點為中心的區間。若以$2R$表示
    區間的長度,則$R$為冪級數的收斂半徑。實際上,冪級數的收斂半徑
    為收斂點的絕對值的上確界。

    \indent 稱$(-R, R)$為冪級數\ref{eq:eq2}的收斂區間。

\end{remark}

\begin{theorem}
    對於冪級數\ref{eq:eq2},若
    \begin{equation}
        \lim_{n \to \infty}\sqrt[n]{\vert a_n \vert} = \rho,
        \label{eq:eq3}
    \end{equation}
    則當
    \begin{enumerate}[label={\rm(\arabic*)}]
        \item $\displaystyle 0 < \rho < +\infty$,冪級數\ref{eq:eq2}的收斂
            半徑為$\displaystyle R = \frac{1}{\rho}$;
        \item $\displaystyle \rho = 0$,冪級數\ref{eq:eq2}的收斂
            半徑為$\displaystyle R = +\infty$;
        \item $\displaystyle \rho = +\infty$,冪級數\ref{eq:eq2}的收斂
            半徑為$\displaystyle R = 0$.
    \end{enumerate}
\end{theorem}

\begin{example}
    討論級數$\displaystyle \sum \frac{x^n}{n^2}$的收斂域。
\end{example}

\begin{example}
    討論級數$\displaystyle \sum \frac{x^n}{n}$的收斂域。
\end{example}

\begin{example}
    討論級數$\displaystyle \sum \frac{x^n}{n!}, \sum n! x^n$的收斂域。
\end{example}

\begin{theorem}{\rm(柯西-阿達馬定理)}
    對於冪級數\ref{eq:eq2},設
    \[
        \rho = \overline{\lim_{n \to \infty}}\sqrt[n]{\vert a_n \vert}
        \]
    則當:
    \begin{enumerate}[label={\rm(\arabic*)}]
        \item $\displaystyle 0 < \rho < +\infty$,冪級數\ref{eq:eq2}的收斂
            半徑為$\displaystyle R = \frac{1}{\rho}$;
        \item $\displaystyle \rho = 0$,冪級數\ref{eq:eq2}的收斂
            半徑為$\displaystyle R = +\infty$;
        \item $\displaystyle \rho = +\infty$,冪級數\ref{eq:eq2}的收斂
            半徑為$\displaystyle R = 0$.
    \end{enumerate}
\end{theorem}
\begin{example}
    討論級數
    \[
        1 + \frac{x}{3} + \frac{x^2}{2^2} + \frac{x^3}{3^2} + 
        \cdots + \frac{x^{2n-1}}{3^{2n-1}} + \frac{x^{2n}}{2^{2n}} + \cdots
        \]
    的收斂域。
\end{example}

\begin{example}
    討論級數
    \[
        \sum_{n=1}^{\infty} \frac{x^{2n}}{n-3^{2n}}
        \]
    的收斂域。
\end{example}

\begin{theorem}
    若冪級數\ref{eq:eq2}的收斂半徑為$\displaystyle R(R>0)$,則冪級數\ref{eq:eq2}在
    它的收斂區間$(-R, R)$內的任一閉區間$[a,b]$上都一致收斂。
\end{theorem}

\begin{theorem}
    若冪級數\ref{eq:eq2}的收斂半徑為$\displaystyle R(R>0)$,且在$x = R $或$x = -R$ 
    時收斂,則冪級數\ref{eq:eq2}在$[0, R]$或$[-R, 0]$上一致收斂。
\end{theorem}

\begin{example}
    討論級數$\displaystyle \sum \frac{(x-1)^n}{2^nn}$的收斂域。
\end{example}

\subsection{冪級數的性質}
\begin{theorem}
    \begin{enumerate}[label = {\rm (\roman*)}]
        \item 冪級數\ref{eq:eq2}的和函數為$(-R, R)$上的連續函數;
        \item 若冪級數\ref{eq:eq2}在左(右)端點上收斂,則其和函數
            在這一端點上右(左)連續。
    \end{enumerate}
\end{theorem}

\ref{eq:eq2}逐項求導,積分為:
\begin{equation}
    a_1 + 2a_2x + 3a_3x^2 + \cdots + n a_n x^{n-1} \cdots + \cdots
    \label{eq:eq4}
\end{equation}
\begin{equation}
    a_0 + 2a_2x + \frac{a_1}{2}x^2 + \cdots +  \frac{a_n}{n+1} x^{n+1} + \cdots
    \label{eq:eq5}
\end{equation}

\begin{theorem}
    冪級數 \ref{eq:eq2}與冪級數\ref{eq:eq4},\ref{eq:eq5}具有相同的
    收斂區間。
\end{theorem}
\begin{theorem}
    設冪級數\ref{eq:eq2}在收斂區間$(-R, R)$上的和函數為$f$,若$x \in (-R, R)$
    則有:
    \begin{enumerate}[label={\rm(\arabic*)}]
        \item $f$在點$x$可導,且
            \[
                f'(x) = \sum_{n=1}^{\infty} n a_n x^{n-1}
                \]
        \item $f$在$0$與$x$之間的區間上可積,且
            \[
                \int_0^xf(t)\,\mathrm{d}t = \sum_{n=0}^{\infty}\frac{a_n}{n+1}x^{n+1}
                \]
    \end{enumerate}
\end{theorem}

\begin{corollary}
    記$f$ 為冪級數\ref{eq:eq2}在收斂區間$(-R, R)$上的和函數,則在$(-R, R)$
    上$f$具有任意階導數,且可以逐項求導數任意次:
    \[
        f'(x) = a_1 + 2a_2x + 3a_3x^2 + \cdots + na_nx^{n-1} + \cdots
    \]
    \[
        f''(x) = 2a_2 + 6a_3x + \cdots + n(n-1)a_nx^{n-2} + \cdots
    \]
    \[
        \cdots\cdots\cdots
    \]
    \[
        f^{(n)}(x) = n!a_n + (n+1)n(n-1)\cdots 2 a_{n+1}x
    \]
\end{corollary}
\begin{corollary}
    記$f$ 為冪級數\ref{eq:eq2}在點$x = 0$ 某鄰域上的和函數,則冪級數
    \ref{eq:eq2}的係屬與$f$在$x = 0$處的各階導數有如下關係:
    \[
        a_0 = f(0), \cdots a_n = \frac{f^{(n)}(0)}{n!}, n=1,2 \cdots
    \]
\end{corollary}

\subsection{冪級數的運算}
\begin{equation}
    \sum_{n=1}^{\infty}a_nx^n
    \label{eq:eq6}
\end{equation}
\begin{equation}
    \sum_{n=1}^{\infty}b_nx^n
    \label{eq:eq7}
\end{equation}

\begin{definition}
    若冪級數\ref{eq:eq6}與冪級數\ref{eq:eq7}在$x=0$點的某鄰域內有相同的
    和函數,則稱這兩個冪級數在該鄰域內相等。
\end{definition}

\begin{theorem}
    若冪級數\ref{eq:eq6}與\ref{eq:eq7}在$x = 0$的某鄰域內相等,則它們同次
    冪的係屬相等。
\end{theorem}

\begin{theorem}
    若冪級數\ref{eq:eq6}與\ref{eq:eq7}的收斂半徑分別是$R_a, R_b$,則有
    \[
        \lambda\sum a_n x^n = \sum \lambda a_nx^n, \vert x \vert < R_a
    \]
    \[
        \sum a_n x^n \pm \sum b_n x^n = \sum (a_n \pm b_n)x^n, \vert x \vert < R
    \]
    \[
        \sum a_n x^n \cdot \sum b_n x^n = \sum c_n x^n, \vert x \vert < R
    \]

    這裡$\displaystyle R = \min\left\{R_a, R_b\right\}, c_n = \sum_{k=1}^{n}a_k b_{n-1}$

\end{theorem}

\begin{example}
    討論幾何級數在收斂域(-1, 1)上的可導,可積分性質。
    \[
        f(x) = \frac{1}{1-x} = 1 + x + x^2 + \cdots + x^n + \cdots
    \]
\end{example}

\begin{example}
    求級數$\displaystyle \sum_{n=1}^{\infty} (-1)^{n-1}n^2x^n$的和函數。
\end{example}
\end{CJK*}
\end{document}
