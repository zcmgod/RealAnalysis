\documentclass[a4paper,12pt]{article}
\usepackage[utf8]{inputenc}
\usepackage{CJKutf8}
\usepackage{multirow}
\usepackage{graphicx}
\usepackage{amsmath}
\usepackage{amssymb}
\usepackage{amsthm}
\usepackage{enumerate}
\usepackage{fancyhdr}
\usepackage{setspace}
\usepackage{enumitem}
\usepackage{lipsum}
\usepackage{indentfirst}

\theoremstyle{theorem}

\addtolength{\parskip}{3pt}
\linespread{1.3} %一倍半行距
\begin{document}
\begin{CJK*}{UTF8}{bsmi}
    \newtheorem{theorem}{定理}[section]
    \newtheorem{lemma}{引理}[section]
    \newtheorem{definition}{定義}[section]
    \newtheorem{example}{例子}[section]
    \newtheorem{corollary}{推論}[section]
    \newtheorem{remark}{注}[section]
    \let\oldref\ref
    \renewcommand{\ref}[1]{\rm{(\oldref{#1})}}
    \renewcommand{\headrulewidth}{0.4pt}
    \renewcommand{\footrulewidth}{0.4pt}

    \title{冪級數}
    \author{武國寧}
    \date{}
    \maketitle

    \section{冪級數}
    \indent 本章將討論由冪函數序列$\left\{a_n(x - x_0)^n\right\}$所產生的函數級數:
    \begin{equation}
        \sum_{n=0}^{\infty}a_n(x - x_0)^n
        \label{eq:eq1}
    \end{equation}
    它稱為冪級數,是一種最簡單的函數項級數。它可以看成是多項式的延伸。

    \indent 下面討論$x_0 = 0$,則\ref{eq:eq1}為:
    \begin{equation}
        \sum_{n=0}^{\infty}a_n x^n
        \label{eq:eq2}
    \end{equation}
    \subsection{冪級數的收斂區間}
    \indent 首先討論冪級數\ref{eq:eq2}的收斂問題。
    \begin{theorem}{\rm(Abel 定理)}
        若冪級數\ref{eq:eq2}在$x = \bar{x} \ne 0$處收斂,則對於滿足
        不等式:$\vert x \vert < \vert \bar{x} \vert$的任何$x$,
        冪級數\ref{eq:eq2}收斂且絕對收斂;若冪級數\ref{eq:eq2}
        在$x = \bar{x} \ne 0$處發散,則對於滿足
        不等式:$\vert x \vert > \vert \bar{x} \vert$的任何$x$,
        冪級數\ref{eq:eq2}發散。
    \end{theorem}
    \begin{remark}
    \indent 冪級數\ref{eq:eq2}的收斂域是以原點為中心的區間。若以$2R$表示
    區間的長度,則$R$為冪級數的收斂半徑。實際上,冪級數的收斂半徑
    為收斂點的絕對值的上確界。

    \indent 稱$(-R, R)$為冪級數\ref{eq:eq2}的收斂區間。

    \end{remark}

    \begin{theorem}
    對於冪級數\ref{eq:eq2},若
    \begin{equation}
        \lim_{n \to \infty}\sqrt[n]{\vert a_n \vert} = \rho,
        \label{eq:eq3}
    \end{equation}
    則當
    \begin{enumerate}[label={\rm(\arabic*)}]
        \item $\displaystyle 0 < \rho < +\infty$,冪級數\ref{eq:eq2}的收斂
            半徑為$\displaystyle R = \frac{1}{\rho}$;
        \item $\displaystyle \rho = 0$,冪級數\ref{eq:eq2}的收斂
            半徑為$\displaystyle R = +\infty$;
        \item $\displaystyle \rho = +\infty$,冪級數\ref{eq:eq2}的收斂
            半徑為$\displaystyle R = 0$.
    \end{enumerate}
    \end{theorem}

    \begin{example}
        討論級數$\displaystyle \sum \frac{x^n}{n^2}$的收斂域。
    \end{example}

\begin{example}
    討論級數$\displaystyle \sum \frac{x^n}{n}$的收斂域。
\end{example}

\begin{example}
    討論級數$\displaystyle \sum \frac{x^n}{n!}, \sum n! x^n$的收斂域。
\end{example}

\begin{theorem}{\rm(柯西-阿達馬定理)}
    對於冪級數\ref{eq:eq2},設
    \[
        \rho = \overline{\lim_{n \to \infty}}\sqrt[n]{\vert a_n \vert}
        \]
    則當:
    \begin{enumerate}[label={\rm(\arabic*)}]
        \item $\displaystyle 0 < \rho < +\infty$,冪級數\ref{eq:eq2}的收斂
            半徑為$\displaystyle R = \frac{1}{\rho}$;
        \item $\displaystyle \rho = 0$,冪級數\ref{eq:eq2}的收斂
            半徑為$\displaystyle R = +\infty$;
        \item $\displaystyle \rho = +\infty$,冪級數\ref{eq:eq2}的收斂
            半徑為$\displaystyle R = 0$.
    \end{enumerate}
\end{theorem}
\begin{example}
    討論級數
    \[
        1 + \frac{x}{3} + \frac{x^2}{2^2} + \frac{x^3}{3^2} + 
        \cdots + \frac{x^{2n-1}}{3^{2n-1}} + \frac{x^{2n}}{2^{2n}} + \cdots
        \]
    的收斂域。
\end{example}

\begin{example}
    討論級數
    \[
        \sum_{n=1}^{\infty} \frac{x^{2n}}{n-3^{2n}}
        \]
    的收斂域。
\end{example}

\begin{theorem}
    若冪級數\ref{eq:eq2}的收斂半徑為$\displaystyle R(R>0)$,則冪級數\ref{eq:eq2}在
    它的收斂區間$(-R, R)$內的任一閉區間$[a,b]$上都一致收斂。
\end{theorem}

\begin{theorem}
    若冪級數\ref{eq:eq2}的收斂半徑為$\displaystyle R(R>0)$,且在$x = R $或$x = -R$ 
    時收斂,則冪級數\ref{eq:eq2}在$[0, R]$或$[-R, 0]$上一致收斂。
\end{theorem}

\begin{example}
    討論級數$\displaystyle \sum \frac{(x-1)^n}{2^nn}$的收斂域。
\end{example}

\subsection{冪級數的性質}
\begin{theorem}
    \begin{enumerate}[label = {\rm (\roman*)}]
        \item 冪級數\ref{eq:eq2}的和函數為$(-R, R)$上的連續函數;
        \item 若冪級數\ref{eq:eq2}在左(右)端點上收斂,則其和函數
            在這一端點上右(左)連續。
    \end{enumerate}
\end{theorem}

\ref{eq:eq2}逐項求導,積分為:
\begin{equation}
    a_1 + 2a_2x + 3a_3x^2 + \cdots + n a_n x^{n-1} \cdots + \cdots
    \label{eq:eq4}
\end{equation}
\begin{equation}
    a_0 + 2a_2x + \frac{a_1}{2}x^2 + \cdots +  \frac{a_n}{n+1} x^{n+1} + \cdots
    \label{eq:eq5}
\end{equation}

\begin{theorem}
    冪級數 \ref{eq:eq2}與冪級數\ref{eq:eq4},\ref{eq:eq5}具有相同的
    收斂區間。
\end{theorem}
\begin{theorem}
    設冪級數\ref{eq:eq2}在收斂區間$(-R, R)$上的和函數為$f$,若$x \in (-R, R)$
    則有:
    \begin{enumerate}[label={\rm(\arabic*)}]
        \item $f$在點$x$可導,且
            \[
                f'(x) = \sum_{n=1}^{\infty} n a_n x^{n-1}
                \]
        \item $f$在$0$與$x$之間的區間上可積,且
            \[
                \int_0^xf(t)\,\mathrm{d}t = \sum_{n=0}^{\infty}\frac{a_n}{n+1}x^{n+1}
                \]
    \end{enumerate}
\end{theorem}

\begin{corollary}
    記$f$ 為冪級數\ref{eq:eq2}在收斂區間$(-R, R)$上的和函數,則在$(-R, R)$
    上$f$具有任意階導數,且可以逐項求導數任意次:
    \[
        f'(x) = a_1 + 2a_2x + 3a_3x^2 + \cdots + na_nx^{n-1} + \cdots
    \]
    \[
        f''(x) = 2a_2 + 6a_3x + \cdots + n(n-1)a_nx^{n-2} + \cdots
    \]
    \[
        \cdots\cdots\cdots
    \]
    \[
        f^{(n)}(x) = n!a_n + (n+1)n(n-1)\cdots 2 a_{n+1}x
    \]
\end{corollary}
\begin{corollary}
    記$f$ 為冪級數\ref{eq:eq2}在點$x = 0$ 某鄰域上的和函數,則冪級數
    \ref{eq:eq2}的係屬與$f$在$x = 0$處的各階導數有如下關係:
    \[
        a_0 = f(0), \cdots a_n = \frac{f^{(n)}(0)}{n!}, n=1,2 \cdots
    \]
\end{corollary}

\subsection{冪級數的運算}
\begin{equation}
    \sum_{n=1}^{\infty}a_nx^n
    \label{eq:eq6}
\end{equation}
\begin{equation}
    \sum_{n=1}^{\infty}b_nx^n
    \label{eq:eq7}
\end{equation}

\begin{definition}
    若冪級數\ref{eq:eq6}與冪級數\ref{eq:eq7}在$x=0$點的某鄰域內有相同的
    和函數,則稱這兩個冪級數在該鄰域內相等。
\end{definition}

\begin{theorem}
    若冪級數\ref{eq:eq6}與\ref{eq:eq7}在$x = 0$的某鄰域內相等,則它們同次
    冪的係屬相等。
\end{theorem}

\begin{theorem}
    若冪級數\ref{eq:eq6}與\ref{eq:eq7}的收斂半徑分別是$R_a, R_b$,則有
    \[
        \lambda\sum a_n x^n = \sum \lambda a_nx^n, \vert x \vert < R_a
    \]
    \[
        \sum a_n x^n \pm \sum b_n x^n = \sum (a_n \pm b_n)x^n, \vert x \vert < R
    \]
    \[
        \sum a_n x^n \cdot \sum b_n x^n = \sum c_n x^n, \vert x \vert < R
    \]

    這裡$\displaystyle R = \min\left\{R_a, R_b\right\}, c_n = \sum_{k=1}^{n}a_k b_{n-1}$

\end{theorem}

\begin{example}
    討論幾何級數在收斂域(-1, 1)上的可導,可積分性質。
    \[
        f(x) = \frac{1}{1-x} = 1 + x + x^2 + \cdots + x^n + \cdots
    \]
\end{example}

\begin{example}
    求級數$\displaystyle \sum_{n=1}^{\infty} (-1)^{n-1}n^2x^n$的和函數。
\end{example}


    \section{函數的冪級數展開}
    \subsection{泰勒級數}
    若函數$f(x)$在$x_0$的某鄰域上存在直到$n+1$階的連續導數,則有:
    \begin{equation}
        f(x) = f(x_0) + f'(x_0)(x - x_0) + \frac{f''(x_0)}{2!}(x - x_0)^2 
               + \cdots + \frac{f^{(n)}(x_0)}{n!}(x - x_0)^n + R_n(x)
               \label{eq:eq8}
    \end{equation}
    這裡$R_n(x)$為拉格朗日型余項:
    \begin{equation}
        R_n(x) = \frac{f^{(n+1)}(\xi)}{(n+1)!}(x - x_0)^{n+1}
        \label{eq:eq9}
    \end{equation}
    如果函數在$x_0$處具有任意階導數,這時稱級數:
    \begin{equation}
         f(x_0) + f'(x_0)(x - x_0) + \frac{f''(x_0)}{2!}(x - x_0)^2 
                + \cdots + \frac{f^{(n)}(x_0)}{n!}(x - x_0)^n + \cdots
        \label{eq:eq10}
    \end{equation}
    為函數在$x_0$處的\textbf{泰勒級數}。對於級數\ref{eq:eq10}在$x_0$點
    的附近能否確切的表達$f$,是本節所要討論的問題。

    \begin{example}
        討論函數 
        \[
            f(x) = \left\{ \begin{array}{ll} 
                e^{-\frac{1}{x^2}}, & x \ne 0 \\
                0,                  & x = 0
            \end{array} \right.
        \]
        在$x = 0$處的泰勒展開式。
    \end{example}

    \begin{theorem}
        設$f(x)$在點$x_0$具有任意階導數,那麼$f(x)$在區間$(x_0-r, x_0+r)$
        上等於它的泰勒級數的和函數的充分必要條件是:對於一切滿足不等式 
        $\displaystyle \vert x - x_0 \vert < r$ 的$x$,有:
        \[
            \lim_{n \to \infty} R_n(x) = 0
        \]
        這裡$R_n(x)$是$f(x)$在$x_0$點的泰勒公式的余項。
    \end{theorem}

    如果$f$能在點$x_0$的某鄰域上等於其泰勒級數的和函數,則稱函數$f$
    在點$x_0$的這一鄰域上可以展開成為泰勒級數,並稱等式
    \begin{equation}
        f(x) = f(x_0) + f'(x_0)(x - x_0) + \frac{f''(x_0)}{2!}(x - x_0)^2 
               + \cdots + \frac{f^{(n)}(x_0)}{n!}(x - x_0)^n + \cdots
               \label{eq:eq11}
    \end{equation}
    的右邊為$f$在$x_0$處的\textbf{泰勒級數展開式},或稱為\textbf{冪級數展開式}。

    \begin{remark}
        函數的泰勒級數展開式是唯一的。
    \end{remark}

    在實際應用上,主要討論函數在$x_0 = 0$處的泰勒展開式,這時級數可以寫作:
    \[
        f(0) + \frac{f'(0)}{1!}x + \frac{f''(0)}{2!}x^2 + \cdots 
             + \frac{f^{(n)}(0)}{n!}x^n + \cdots 
    \]
    稱為\textbf{麥克勞林級數}。

    \begin{theorem}{\rm (Cauchy型余項)}
        設$f(x)$在$(x_0-r, x_0+R)$上任意階可導,則有:
        \[
            f(x) = \sum_{k=0}^{n} \frac{f^{(k)}(x_0)}{k!}(x - x_0)^k 
                   + R_n(x), x \in (x_0-r, x_0+r)
        \]
        其中,
        \begin{equation}
            R_n(x) = \frac{1}{n!}\int_{x_0}^{x} f^{(n+1)}(t)(x - t)^n\,
                     \mathrm{d}t
        \end{equation}
    \end{theorem}

    \begin{proof}
        由表達式
        \[
            r_n(x) = f(x) - \sum_{k=0}^{n}\frac{f^{(k)}(x_0)}{k!}(x - x_0)^k
        \]
         出發,逐次對兩段求導數,可得到:
        \[
            r'_n(x) = f'(x) - \sum_{k=1}^{n}\frac{f^{(k)}(x_0)}{(k-1)!}(x - x_0)^{k-1}
        \]
        \[
            r''_n(x) = f''(x) - \sum_{k=2}^{n}\frac{f^{(k)}(x_0)}{(k-2)!}(x - x_0)^{k-2}
        \]
        \[
            \cdots
        \]
        \[
            r_n^{(n)}(x) = f^{(n)}(x) - f^{(n)}(x_0)
        \]
        \[
            r_n^{(n+1)}(x) = f^{(n+1)}(x) 
        \]
        令$x = x_0$, 有
        \[
            r_n(x_0) = r'_n(x_0) = r''_n(x_0) = \cdots = r_n^{(n)}(x_0) = 0
        \]
        逐次積分,有
        \[
            \begin{split}
                r_n(x) & = r_n(x) - r_n(x_0) = \int_{x_0}^{x}r'(t)\,\mathrm{d}t \\
                       & = \int_{x_0}^{x}r'(t)\,\mathrm{d}(t - x) \\
                       & = \int_{x_0}^{x}r''(t)(x - t)\,\mathrm{d}t \\
                       & \cdots \\
                       & = \frac{1}{n!}\int_{x_0}^{x}r_n^{(n+1)}(t)(x - t)^n\, 
                           \mathrm{d}t \\
                       & = \frac{1}{n!}\int_{x_0}^{x}f^{(n+1)}(t)(x - t)^n\, 
                           \mathrm{d}t 
            \end{split}
        \]
    \end{proof}
    
    \begin{remark}
        \begin{enumerate}[label={\rm(\arabic*)}]
            \item 為拉格朗日型余項可以有柯西型余項的到,需要利用積分第一中值定理。
            \item 若果將函數$\displaystyle f^{(n+1)}(t)(x - t)^n$看成一個函數,
                利用積分第一中值定理,則有:
                \begin{equation}
                    \begin{split}
                        R_n(x) & = \frac{f^{(n+1)}(\xi)(x - \xi)^n}{n!} 
                    \int_{x_0}^{x}\, \mathrm{d}t (\xi \in (x_0, x)) \\
                        & = \frac{f^{(n+1)}(x_0 + \theta(x - x_0))}{n!} 
                            (1 - \theta)^n(x - x_0)^{n+1}, 0 \le \theta \le 1.
                    \end{split}
                \end{equation}
        \end{enumerate}
    \end{remark}

    \subsection{初等函數的冪級數展開式}
    \begin{example}
        求$k$次多項式函數
        \[
            f(x) = c_0 + c_1 x + c_2 x^2 + \cdots + c_k x^k
        \]
        的展開式。
    \end{example}

    \begin{example}
        求函數$\displaystyle f(x) = e^x$的展開式。
    \end{example}

    \begin{example}
        求函數$\displaystyle f(x) = \sin x$的展開式。
    \end{example}

    \begin{example}
        求函數$\displaystyle f(x) = \ln (1+x)$的展開式。
        提示:
        \begin{enumerate}[label={\rm(\arabic*)}]
            \item 當$0 \le x \le 1$時使用\textbf{拉格朗日型余項},有
                \[
                    \begin{split}
                        \left| R_n(x) \right| &  = 
                        \left| \frac{1}{(n+1)!}(-1)^n 
                        \frac{n!}{(1+\xi)^{n+1}}x^{n+1} \right| \\
                        & = \left|\frac{(-1)^n}{n+1}\left(\frac{x}
                        {(1+\xi)}\right)^{n+1}\right| \\
                        & \le \frac{1}{n+1} \to 0 (n \to \infty)
                    \end{split}
                \]
            \item 當$-1 < x < 0$的情形,拉格朗日型余項不易估計,改用柯西型
                余項,有
                \[
                    \begin{split}
                        \left| R_n(x) \right| & = \left| \frac{1}{n!}(-1)^n 
                        \frac{n!}{(1+\theta x)^{n+1}}(1-\theta)^n x^{n+1}
                        \right| \\
                        & = \frac{1}{1+\theta x}\left(\frac{1-\theta}
                        {1+\theta x}\right)^n\vert x \vert^{n+1}, 
                        0 \le x \le 1.
                    \end{split}
                \]
                因為$-1 < x < 0$,所以有$1 - \theta \le 1 + \theta x$.
                即有$\displaystyle 0 \le \frac{|x|^{n+1}}{1-|x|} \to 0 
                (n \to \infty)$.
        \end{enumerate}
    \end{example}
    \begin{example}
        討論二項式函數$f(x) = (1+x)^{\alpha}$的展開式。
        關於收斂區間的討論(其推導過程參見菲赫金哥爾茨著《微積分教程》
        第二卷第二份冊):
        \begin{enumerate}[label={\rm(\arabic*)}]
            \item 當$\alpha \le -1$時,收斂域為$(-1,1)$;
            \item 當$-1 < \alpha < 0$時,收斂域為$(-1,1]$;
            \item 當$\alpha > 0$時,收斂域為$[-1, 1]$;
        \end{enumerate}
    \end{example}

    \begin{example}
        利用$f(x) = (1+x)^{\alpha}$的展開式,得到
        \begin{equation}
            \frac{1}{1+x^2} = 1 - x^2 + x^4 + \cdots + (-1)^nx^{2n} + 
            \cdots, (-1, 1).
        \end{equation}
        \begin{equation}
            \frac{1}{\sqrt{1-x^2}} = 1 + \frac{1}{2}x^2 + \frac{1 \cdot 3}
            {2 \cdot 4}x^4 + \frac{1 \cdot 3 \cdot 5}{2 \cdot 4 \cdot 6}
            x^6 + \cdots, (-1, 1)
        \end{equation}
        \begin{equation}
            \begin{split}
                \emph{arctan}x & = \int_0^x\frac{\mathrm{d}t}{1+t^2}\\
                & = x - \frac{x^3}{3} + \frac{x^5}{5} + \cdots + 
                (-1)^n\frac{x^{2n+1}}{2n+1} + \cdots, [-1, 1]
            \end{split}
        \end{equation}
        \begin{equation}
            \begin{split}
                \emph{arcsin}x & = \int_0^x\frac{\mathrm{d}t}{\sqrt{1-t^2}}\\
                & = x + \frac{1}{2}\frac{x^3}{3} + \frac{1 \cdot 3}{2 \cdot 4}
                \frac{x^5}{5} + \frac{1 \cdot 3 \cdot 5}{2 \cdot 4 \cdot 6}\frac{x^7}{7} 
                + \cdots, [-1, 1]
            \end{split}
        \end{equation}
    \end{example}
    
    \begin{example}
        求函數$\displaystyle f(x) = (1-x)\ln(1-x)$在$x=0$點的展開式。
    \end{example}
    \begin{example}
        用間接方法求非初等函數
        \[
            F(x) = \int_0^x e^{-t^2}\, \mathrm{d}t
        \]
        的冪級數展開式。
    \end{example}

\end{CJK*}
\end{document}
