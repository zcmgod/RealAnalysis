\documentclass[a4paper,12pt]{article}
\usepackage{CJKutf8}
\usepackage{multirow}
\usepackage{graphicx}
\usepackage{amsmath}
\usepackage{amsfonts}
\usepackage{amssymb}
\usepackage{amsthm}
\usepackage{enumerate}
\usepackage{fancyhdr}
\usepackage{setspace}
\usepackage{bm}
%\pagestyle{fancy}
%\lhead{}
%\rhead{\bfseries Integral on lines and surfaces}
%\lfoot{Calculus}
%\cfoot{China University of Petroleum-Beijing}
%\rfoot{\thepage}
%\renewcommand{\headrulewidth}{0.4pt}
%\renewcommand{\footrulewidth}{0.4pt}
%%%% 段落首行缩进两个字 %%%%
\makeatletter
\let\@afterindentfalse\@afterindenttrue
\@afterindenttrue
\makeatother
\setlength{\parindent}{2em}  %中文缩进两个汉字位

\newcounter{solution}
\newcommand\TheSolution{%
  \textbf{Solution:}\\%
}

%%%% 下面的命令重定义页面边距,使其符合中文刊物习惯 %%%%
\addtolength{\topmargin}{-54pt}
\setlength{\oddsidemargin}{0.63cm}  % 3.17cm - 1 inch
\setlength{\evensidemargin}{\oddsidemargin}
\setlength{\textwidth}{14.66cm}
\setlength{\textheight}{24.00cm}    % 24.62

%%%% 下面的命令设置行间距与段落间距 %%%%
\linespread{1.4}
% \setlength{\parskip}{1ex}
\setlength{\parskip}{0.5\baselineskip}

%%%% 定理类环境的定义 %%%%
\newtheorem{example}{Example}             % 整体编号
\newtheorem{algorithm}{算法}
\newtheorem{theorem}{Theorem}[section]  % 按 section 编号
\newtheorem{definition}{Definition}
\newtheorem{axiom}{公理}
\newtheorem{property}{性质}
\newtheorem{proposition}{Proposition}
\newtheorem{lemma}{引理}
\newtheorem{corollary}{Corollary}
\newtheorem{remark}{注解}
\newtheorem{condition}{条件}
\newtheorem{conclusion}{结论}
\newtheorem{assumption}{假设}

%%%% 重定义 %%%%
\renewcommand{\contentsname}{目录}  % 将Contents改为目录
\renewcommand{\abstractname}{摘要}  % 将Abstract改为摘要
\renewcommand{\refname}{参考文献}   % 将References改为参考文献
\renewcommand{\indexname}{索引}
%\renewcommand{\figurenamGe}{图}
\renewcommand{\tablename}{表}
\renewcommand{\appendixname}{附录}
\renewcommand{\algorithm}{算法}
\begin{document}
\begin{CJK}{UTF8}{gbsn}
\title{Integrals Depending on a Parameter}
\author{Guoning Wu\footnote{Email: wuguoning@163.com}\\[2ex]
 China University of Petroleum-Beijing\\[2ex]}
\date{2017.9}
\maketitle

\section{Proper Integrals Depending on a Parameter}
    \subsection{The Concept of an Integral Depending on a Parameter}
    An integral depending on a parameter is a function of the form 
    \begin{equation}
        F(t) = \int_{E_t} f(x,t)\, \mathrm{d}t,
        \label{eq:eq1}
    \end{equation}
    where $t$ plays the role of a parameter ranging over a set $T$, and 
    to each value $t \in T$ there corresponding a set $E_t$  and a function 
    $\varphi_t(x) = f(x,t)$ that is integrable over $E_t$ in the proper or improper 
    sense. The nature of the set $T$ may be quite varied, but of course 
    the most important cases occurs when $T$ is a subset of $\mathbb{R},
    \mathbb{C}, \mathbb{R}^n, \mathbb{C}^n$.

    If the integral (Eq.~\ref{eq:eq1}) is a proper integral for each 
    value of the parameter $t \in T$, we say that the function in Eq.
    ~\ref{eq:eq1} is a proper integral depending on a parameter.

    But if the integral in Eq.~\ref{eq:eq1} exists only as an improper 
    integral for some or all of the value of $t \in T$, we usually call 
    $F$  an improper integral depending on a parameter.

    \subsection{Continuity of an Integral Depending on a Parameter}
    \begin{proposition}
        Let $P = \left\{\left(x,y\right)\in \mathbb{R}^2 \vert a \le x \le b,
        c \le y \le d\right\}$ be a rectangle on the plane $\mathbb{R}^2$.
        If the function $f: P \to \mathbb{R}$ is continuous, that is, if 
        $f \in C(P,\mathbb{R})$, then the function 
        \begin{equation}
            F(y) = \int_a^bf(x,y)\,\mathrm{d}x
            \label{eq:eq2}
        \end{equation}
        is continuous at every point $y \in [c,d]$.
    \end{proposition}

    \begin{example}
        Find the limit 
        \[
            \lim_{a \to 0} \int_0^1\frac{dx}{1+x^2\cos ax}
        \]
    \end{example}

    \begin{proposition}
        Suppose $f(x,y) \in C\left([a,b]\times[c,d]\right)$, then 
        \[
            \int_c^d\,\mathrm{d}y\int_a^bf(x,y)\,\mathrm{d}x  = 
            \int_a^b\,\mathrm{d}x\int_c^df(x,y)\,\mathrm{d}y.
        \]
    \end{proposition}

    \begin{example}
        Find the value 
        \[
            I = \int_0^1\frac{x^b-x^a}{\ln x}\, \mathrm{d}x, for\quad b > a > 0
            \]
    \end{example}

    \subsection{Differentiation of an Integral Depending on a Parameter}
    \begin{proposition}
        If the function $f: P \to \mathbb{R}$ is continuous and has a continuous 
        partial derivative with respect to $y$ on the rectangle $P = \left\{ \left(x,y\right)
        \in \mathbb{R}^2 \vert a \le x \le b, c \le y \le d\right\}$, then the integral 
        of Eq.~\ref{eq:eq2} belongs to $C^{(1)}\left([c,d],\mathbb{R}\right)$, and 
        \[
            F'(y) = \int_a^b\frac{\partial f}{\partial y}(x,y)\, \mathrm{d}x.
        \]
    \end{proposition}

    \begin{example}
        The complete elliptic integrals
        \[
            E(k) = \int_0^{\frac{\pi}{2}}\sqrt{1-k^2\sin^2\varphi}\, \mathrm{d}\varphi,
            K(k) = \int_0^{\frac{\pi}{2}}\frac{\mathrm{d}\varphi}{\sqrt{1-k^2\sin^2\varphi}}
        \]
    as functions of the parameter $k, 0 < k < 1$, called the modulus of the 
    corresponding elliptic integral, are connected by the relations
        \[
            \frac{\mathrm{d}E}{\mathrm{d}k} = \frac{E - K}{k},
            \frac{\mathrm{d}K}{\mathrm{d}k} = \frac{E}{k(1-k^2)} - \frac{K}{k}.
        \]
    \end{example}

    \begin{proposition}
        Suppose the function $f: P \to \mathbb{R}$ is continuous and has a 
        continuous partial derivative  $\frac{\partial f}{\partial y}$ on the rectangle $P = \left\{ \left(x,y\right)
        \in \mathbb{R}^2 \vert a \le x \le b, c \le y \le d\right\}$, further suppose 
        $\alpha(y), \beta(y)$ are continuously differentiable functions on 
        $[c,d]$ whose values lie in $[a,b]$ for every $y \in [c,d]$. Then the integral 
        \[
            F(y) = \int_{\alpha(y)}^{\beta(y)}f(x,y)\, \mathrm{d}x
        \]
        is defined for every $y \in [c,d]$ and belongs to $C^{(1)} \left([c,d]\right)$
        , and the following formula holds;
        \[
            F'(y) = f(\beta(y),y)\cdot\beta'(y) - f(\alpha(y),y)\cdot\alpha'(y) 
            + \int_{\alpha(y)}^{\beta(y)}\frac{\partial f}{\partial y}(x,y)\,
            \mathrm{d}x.
        \]
    \end{proposition}

    \begin{example}
        Let 
        \[
            F_n(x) = \frac{1}{(n-1)!}\int_0^x(x-t)^{n-1}f(t)\, \mathrm{d}t,
        \]
        where $n \in N$ and $f$ is a function that is continuous on the interval 
        of integration. Let us verify that $F_n^{n}(x) = f(x)$.
    \end{example}

    \subsection{Integration of an Integral Depending on a Parameter}
    \begin{proposition}
        If the function $f: P \to \mathbb{R}$ is continuous in the rectangle 
        $P = \left\{\left(x,y\right) \in \mathbb{R}^2 \vert a \le x \le b and 
        c \le y \le d \right\}$ , then the integral Eq.~\ref{eq:eq2} is integrable over 
        the closed interval $[c,d]$ and the following equality holds:
        \begin{equation}
            \int_c^d\int_a^bf(x,y)\,\mathrm{d}x\,\mathrm{d}y =
            \int_a^b\int_c^df(x,y)\,\mathrm{d}y\,\mathrm{d}x
            \label{eq:eq3}
        \end{equation}
    \end{proposition}

    \section{Improper Integrals Depending on a Parameter}
    \subsection{Uniform Convergence of an Improper Integral With Respect 
    to a Parameter}
    \paragraph{\textbf{a. Basic Definition and Examples}}
    Suppose that the improper integral 
    \begin{equation}
        F(y) = \int_a^{\omega}f(x,y)\,\mathrm{d}x
        \label{eq:eq4}
    \end{equation}
    over the interval $[a,\omega]$ converges for each value $y \in Y$. For 
    definiteness we shall assume that the integral Eq.~\ref{eq:eq4}
    has only one singularity and that it involves the upper limit 
    of the integration (that is, either $\omega = +\infty$ or the function $f$ is unbounded 
    as a function of $x$ in a neighbourhood of $\omega$.)

    \begin{definition}
        We say that the improper integral Eq.~\ref{eq:eq4} depending on 
        the parameter $y \in Y$ converges uniformly on the set $E \subset Y$
        if for every $\epsilon > 0$ there exists a neighborhood $U_{[a,\omega[}(\omega)$
        of $\omega$ in the set $[a,\omega[$ such that the estimate
        \begin{equation}
            \left | \int_b^\omega f(x,y)\, \mathrm{d}x \right | < \epsilon
            \label{eq:eq5}
        \end{equation}
        for the remainder of the integral Eq.\ref{eq:eq4} holds for every 
        $b \in U_{[a,\omega[}(\omega)$ and every $y \in E$.
    \end{definition}

    If we introduce the notation 
    \begin{equation}
        F_b(y) = \int_a^bf(x,y)\, \mathrm{d}x
        \label{eq:eq6}
    \end{equation}
    for a proper integral approximating the improper integral of 
    Eq.\ref{eq:eq4}, the basic definition of this section can be restated as 
    in a different form equivalent to the previous one:
    uniform convergence of the integral of Eq.~\ref{eq:eq4} on the set 
    $E \subset Y$ by definition means that 
    \begin{equation}
        F_b(y) \rightrightarrows F(y) \text{ on } E \text{ as } b \to \omega, b \in [a,\omega[
    \end{equation}

    \begin{example}
        The integral 
        \[
            \int_1^{+\infty} \frac{\mathrm{d}x}{x^2+y^2}
        \]
        converges uniformly on the entire set $\mathbb{R}$ of values of the 
        parameter $y \in \mathbb{R}$.
    \end{example}

    \begin{example}
        The integral 
        \[
            \int_0^{+\infty} e^{-xy}\, \mathrm{d}x
        \]
    converges only when $y > 0$. Moreover it converges uniformly on every 
        set $\left\{y \in \mathbb{R} \vert y \ge y_0 \ge 0 \right\}.$
    \end{example}

    \begin{example}
        Let us show that each of the integrals 
        \[
            \Phi(x) = \int_0^{+\infty} x^{\alpha}y^{\alpha+\beta+1}e^{-(1+x)y}\,\mathrm{d}y
        \]
        \[
            F(y) = \int_0^{+\infty} x^{\alpha}y^{\alpha+\beta+1}e^{-(1+x)y}\,\mathrm{d}x
        \]
    in which $\alpha,\beta$ are fixed positive numbers, converges uniformly on the set 
    of non-negative values of the parameter.
    \end{example}

    \paragraph{b. The Cauchy Criterion for Uniform Convergence of an Integral}
    \begin{proposition}{\textbf{Cauchy Criterion.}}
        A necessary and sufficient condition for the improper integral of 
        Eq.~\ref{eq:eq4} depending on parameter $y \in Y$ to converge uniformly 
        on a set $E \subset Y$ is that for every $\epsilon > 0$ there exist a neighborhood 
        $U_{[a,\omega[}$ of the point $\omega$ such that 
        \[
           \left| \int_{b_1}^{b_2}f(x,y)\right| < \epsilon
        \]
    for every $b_1,b_2 \in U_{[a,\omega[}$ and every $y \in E$.
    \end{proposition}

    \begin{corollary}
        If the function $f$ in the integral of Eq.~\ref{eq:eq4} is continuous 
        on the set $[a, \omega[ \times [c, d]$ and the integral of Eq.~\ref{eq:eq4} 
        converges for every $y \in ]c,d[$ but diverges for $y = c$ or $y = d$,
        then it converges non-uniformly on the interval $]c,d[$ and also on 
        any set $E \subset ]c,d[$ whose closure contains the point of divergence.
    \end{corollary}

    \begin{example}
        The integral 
        \[
            \int_0^{+\infty}e^{-tx^2}\, \mathrm{d}x
            \]
        converges for $t > 0$ and diverges at $t = 0$, hence it demonstrably converges 
        non-uniformly on every set of positive numbers having $0$ as a limit point.
    \end{example}

    \paragraph{\textbf{c. Sufficient Conditions for Uniform Convergence of 
    an Improper Integral Depending on a Parameter}}

    \begin{proposition}{The Weierstrass test.}
        Suppose the functions $f(x,y)$ and $g(x,y)$ are integrable with respect to $x$
        on every closed interval $[a,b] \subset [a,\omega[$ for each value of $y \in Y$.
        
        If the inequality $\left | f(x,y) \right | \le g(x,y) $ holds for each value 
        of $y \in Y$ and every $x \in [a,\omega[$ and the integral 
        \[
            \int_a^{\omega}g(x,y)\,\mathrm{d}x
        \]
        converges uniformly on $Y$, then the integral 
        \[
            \int_0^{\omega}f(x,y)\,\mathrm{d}x
        \]
        converges absolutely for each $y \in Y$ and uniformly on $Y$.
    \end{proposition}

    The most frequently encountered case of Proposition 2 occurs when the function 
    $g$ is independent of the parameter $y$. It is this case in which Proposition 2 
    is usually called the Weierstrass M-test for uniform convergence of an integral.

    \begin{example}
        The integral 
        \[
            \int_0^{\infty}\frac{\cos \alpha x}{1 + x^2}\, \mathrm{d}x
        \]
        converges uniformly on the whole set $\mathbb{R}$ of the parameter $\alpha$,
        since $\left |\frac{\cos \alpha x}{1 + x^2}\right | \le \frac{1}{1 + x^2}$, and the integral 
        $\int_0^{\infty}\frac{\mathrm{d}x}{1 + x^2}$ converges.
    \end{example}

    \begin{proposition}{(\emph{Abel-Dirichlet test.})}
        Assume that the function $f(x,y)$ and $g(x,y)$ are integrable with respect to 
        $x$ at each $y \in Y$ on every closed interval $[a,b]\subset[a,\omega[.$

        A sufficient condition for uniform convergence of the integral 
        \[
            \int_a^{\omega}\left(f\cdot g\right)\, \mathrm{d}x
        \]
        on the set $Y$ is that one of the following two pairs of conditions holds:

        \emph{1-1)} either there exists a constant $M \in \mathbb{R}$ such that 
        \[
            \left |\int_a^bf(x,y)\, \mathrm{d}x \right | < M
        \]
        for any $b \in [a, \omega[$ and any $y \in Y$ and 

        \emph{1-2)} for each $y \in Y$ the function $g(x,y)$ is monotonic with respect to $x$ on 
        the interval $[a, \omega[$ and $g(x,y)\rightrightarrows 0$ on $Y$ as $x \to \omega$,
        $x \in [a,\omega[$, or 

        \emph{2-1)} the integral 
        \[
            \int_a^{\omega}f(x,y)\, \mathrm{d}x
            \]
        converges uniformly on the set $Y$ and 

        \emph{2-2)} for each $y \in Y$ the function $g(x,y)$ is monotonic with respect 
        to $x$ on the interval $[a, \omega[$ and there exists a constant $M \in \mathbb{R}$
        such that 
        \[
            \left |g(x,y)\right| < M
            \]
        for every $x \in [a, \omega[$ and every $y \in Y$.
    \end{proposition}
    Applying the second mean-value theorem for the integral, we have
    \[
        \int_{b_1}^{b_2} \left(f\cdot g\right)(x,y)\, \mathrm{d}x = g(b_1,y)
        \int_{b_1}^{\xi} f(x,y)\, \mathrm{d}x + g(b_2,y)\int_{\xi}^{b_2}f(x,y)\, \mathrm{d}x
        \]

    \begin{example}
        The integral 
        \[
            \int_0^{\infty}\frac{\sin x}{x}e^{-xy}\, \mathrm{d}x
            \]
        converges uniformly on the set $\left\{y \in \mathbb{R} \vert y \ge 0 \right\}$.
    \end{example}

    \begin{example}
        The integral 
        \[
            \int_0^{\infty} \frac{\sin xy}{x}\, \mathrm{d}x
            \]
        converges uniformly on the set $\left\{y \in \mathbb{R} \vert y \ge y_0 > 0
        \right\}$ and not uniformly convergence on the set $\left\{y \in \mathbb{R} \vert y > 0
        \right\}$ 
    \end{example}

    \begin{example}
        \emph{The integrali} $\displaystyle \int_0^{+\infty} \frac{\cos x^2}{x^p}\, \mathrm{d}x$
        \emph{converges uniformly on each} $p \in [\alpha,\beta]\subset(-1,1)$.
    \end{example}

    \section{Limiting Passage under the Sign of an Improper Integral and 
    Continuity of an Improper Integral Depending on a Parameter}
    \begin{proposition}
        Let $f(x,y)$ be a family of functions depending on a parameter $y \in Y$ that 
        are integrable, possibly in the improper sense, on the interval $a \le x \le 
        \omega$, and let $\mathcal{B}_Y$ be a base in $Y$.
        
        If 

        a) for every $b \in [a,\omega[$ 
        \[
            f(x,y)\rightrightarrows \varphi(x) \text{ on } [a,b]
            \text{ over the base } \mathcal{B}_Y,
            \]
        b) the integral $\displaystyle \int_a^{\omega}f(x,y)\,\mathrm{d}x$
        converges uniformly on $Y$,
        then the limit function $\varphi$is improperly integrable on $[a,\omega[$
        and the following equality holds:
        \[
            \lim_{\mathcal{B}_Y}\int_a^{\omega}f(x,y)\, \mathrm{d}x = \int_a^{\omega}\varphi 
            \, \mathrm{d}x.
            \]
    \end{proposition}

    \begin{proposition}
        If 

        a) the function $f(x,y)$ is continuous on the set 

        $\displaystyle \left\{
            (x,y) \in \mathbb{R}^2 \vert a \le x <\omega, c \le y \le d \right\}$
        and 

        b) the integral $F(y) = \int_a^{\omega}f(x,y)\, \mathrm{d}x$ converges 
        uniformly on $[c,d]$,

        then the function $F(y)$ is continuous on $[c,d].$
    \end{proposition}

    \begin{proposition}
        Suppose $f(x,y)$ is continuous on $[a,+\infty) \times [c,d]$, and 
        the integral $\int_a^{\infty}f(x,y)\, \mathrm{d}x$ converges uniformly on 
        $[c,d]$, then we have 
        \[
            \int_c^d\,\mathrm{d}y\int_a^{+\infty}f(x,y)\, \mathrm{d}x = 
            \int_a^{+\infty}\,\mathrm{d}x\int_c^df(x,y)\,\mathrm{d}y
            \]
    \end{proposition}

    \begin{proposition}
        Suppose $f(x,y), f_y(x,y)$ are continuous on $[a,+\infty)\times[c,d]$, 
        for each $y \in [c,d]$ the integral $\int_a^{+\infty}f(x,y)\, \mathrm{d}x$ 
        converges. Furthermore the integral $\int_a^{+\infty}f_y(x,y)\,\mathrm{d}x$
        is uniformly converges. Then we have 
        \[
            \frac{d}{dy}\int_a^{+\infty}f(x,y)\,\mathrm{d}x = 
            \int_a^{+\infty}\frac{\partial}{\partial y}f(x,y)\,\mathrm{d}x
            \]
    \end{proposition}

    \section{The Eulerian Integrals}
    In this section and the next we shall illustrate the application of the theory 
    developed above to some specific integrals of importance in analysis that 
    depend on a parameter.

    Following Legendre, we define the Eulerian integrals of first and second 
    kinds respectively as the two special functions that follow:

    \begin{equation}
        B(\alpha, \beta) = \int_0^1x^{\alpha - 1}(1-x)^{\beta - 1}\, \mathrm{d}x
    \end{equation}
    \begin{equation}
        \Gamma(\alpha) = \int_0^{+\infty} x^{\alpha - 1} e^{-x}\, \mathrm{d}x.
    \end{equation}

    The first of these is called the beta function and the second the gamma function.

    \subsection{The Beta Function}
    \paragraph{\rm \textbf{a. Domain of Definition}}
    A necessary and sufficient condition for the convergence of the integral of the 
    beta function at the lower limit is that $\alpha > 0$. Similarly, convergence at 1
    occurs if and only if $\beta > 0.$ Thus the beta function is defined when both
    of the following conditions hold simultaneously:
    \[
        \alpha > 0 \text{ and } \beta > 0
        \]

    \paragraph{\rm \textbf{b. Symmetry}}
    We can verify that:
    \[
        B(\alpha, \beta) = B(\beta, \alpha)
        \]
    
    \

    \paragraph{\rm \textbf{c. The Reduction Formula}}
    If $\alpha > 1$, the following equality holds:
    \[
        B(\alpha, \beta) = \frac{\alpha - 1}{\alpha + \beta - 1}B(\alpha - 1, \beta)
        \]

    We can now write the reduction form:
    \[
        B(\alpha, \beta) = \frac{\alpha - 1}{\alpha + \beta - 1}B(\alpha, \beta - 1)
        \]

    It can be seen immediately from the definition of the beta function that 
    $B(\alpha, 1) = \frac{1}{\alpha}$, and so for $n \in \mathbb{N}$ we obtain 
    \begin{equation}
        \begin{split}
            B(\alpha, n) & = \frac{n-1}{\alpha+n-1}\cdot\frac{n-2}{\alpha+n-2}\cdots
        \frac{n-(n-1)}{\alpha+n-(n-1)}B(\alpha,1)\\
            & = \frac{(n-1)!}{\alpha (\alpha+1)\cdots(\alpha+n-1)}.
        \end{split}
    \end{equation}

    In particular, for $m, n \in \mathbb{N}$
    \begin{equation}
        B(m,n) = \frac{(m-1)!(n-1)!}{(m+n-1)!}
    \end{equation}

    \paragraph{\rm \textbf{d. Other forms of Representation of the Beta Function}}

    (1) One form for the beta function is 
    \[
        B(\alpha, \beta) = 2\int_0^{\frac{\pi}{2}} \cos^{2p-1}\phi\sin^{2q-1} \phi
        \, \mathrm{d}\phi
        \]
    \[
        B\left(\frac{1}{2}, \frac{1}{2}\right) = \pi
        \]

    (2) The other form for the beta function is 
    \[
        B(\alpha, \beta) = \int_0^{+\infty} \frac{t^{\beta -1}}{(1+t)^{\alpha + \beta}}
        \, \mathrm{d}t
        \]

    \section{The Gamma Function}

    \paragraph{\rm \textbf{a. Domain of the Definition}}
    The Gamma Function is:
    \[
        \Gamma(\alpha) = \int_0^{+\infty} x^{\alpha -1}e^{-x}\, \mathrm{d}x
        \]
    It can be seen from the definition that the integral defining the gamma 
    function converges at zero only for $\alpha > 0$, while it converges at infinity 
    for all values of $\alpha \in \mathbb{R}$, due to the presence of the rapidly decreasing 
    factor $e^{-x}$. Thus the gamma function is defined for $\alpha > 0$.

    \paragraph{\rm \textbf{b. Smoothness and the Formula for the Derivatives}}
    The gamma function is infinitely differentiable, and 
    \begin{equation}
        \Gamma^{(n)}(\alpha) = \int_0^{+\infty}x^{\alpha - 1} \ln^nx e^{-x}\, \mathrm{d}x
        \label{eq:dgamma}
    \end{equation}

    \paragraph{\rm \textbf{c. The Reduction Formula}}
    The relation 
    \[
        \Gamma(\alpha + 1) = \alpha\Gamma(\alpha)
        \]
    holds. It is known as the reduction formula for the gamma function.

    Since $\displaystyle \Gamma(1) = 1$, we conclude that for $n \in \mathbb{N}$
    \[
        \Gamma(n+1) = n!
        \]
    Thus the gamma function turns out to be closely connected with the 
    number-theoretic function $n!$.

    \paragraph{\rm \textbf{d. The Euler-Gauss Formula}}
    This is usually given to the following equality:
    \begin{equation}
        \Gamma(\alpha) = \lim_{n \to \infty} n^{\alpha}\frac{(n-1)!}{\alpha(\alpha+1)\cdots(\alpha+n-1)}
    \end{equation}

    \paragraph{\rm \textbf{e. The Complement Formula}}
    For $0 < \alpha < 1$ the values $\alpha$ and $1 - \alpha$ of the argument of the gamma
    function are mutually complementary, so that the equality 
    \begin{equation}
        \Gamma(\alpha)\Gamma(1-\alpha) = \frac{\pi}{\sin \pi\alpha} (0 < \alpha < 1)
        \label{eq:complement eq}
    \end{equation}

    It follows in particular from EQ~\ref{eq:complement eq} that 
    \[
        \Gamma\left(\frac{1}{2}\right) = \sqrt{\pi}
        \]
    We observe that 
    \[
        \Gamma\left(\frac{1}{2}\right) = \int_0^{+\infty}x^{-\frac{1}{2}}e^{-x}\, \mathrm{d}x
         = 2\int_0^{+\infty}e^{-u^2}\,\mathrm{d}u = \sqrt{\pi}
         \]

     \paragraph{\rm \textbf{f. Connection Between the Beta and Gamma Function}}
     The connection between the beta and gamma function is 
     \begin{equation}
         B\left(\alpha, \beta\right) = \frac{\Gamma(\alpha)\Gamma(\beta)}{\Gamma(\alpha + \beta)}.
     \end{equation}

     \begin{example}
         Find the result of 
         \[
             I = \int_0^{\frac{\pi}{2}}\sin^6x\cos^4x\, \mathrm{d}x
             \]
     \end{example}

     \begin{example}
         Find the result of 
         \[
             \int_0^1x^8 \sqrt{1-x^3}\, \mathrm{d}x
             \]
     \end{example}

     \begin{example}
         suppose $\alpha > -1$, find the results of the following integrals
         \[
             \int_0^{\frac{\pi}{2}}\sin^{\alpha}x\, \mathrm{d}x  = 
             \int_0^{\frac{\pi}{2}}\cos^{\alpha}x\, \mathrm{d}x 
             \]
         Furthermore, find the volume of the n-dimensional sphere of the form
         \[
             B_n = \left\{(x_1, x_2, \cdots ,x_n) \vert x_1^2+x_2^2+\cdots+x_n^2 \le R^2\right\}
             \]
     \end{example}
\end{CJK}
\end{document}
